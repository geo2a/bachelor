\chapter{Заключение}

\section{Основные результаты работы} 
  
  Автором были получены следующие \textbf{основные результаты}:

  \begin{enumerate}
    
    \item Разработан метод построения монадических комбинаторов парсеров, основанный 
    на трансфорерах монад. Исходный код библиотеки доступен в репозитории на GitHub~\cite{mdParse}. Библиотечные парсеры доступны в файле \lstinline{src/Parsers/Parsers.hs}. 

    \item Разработан прототип метода построения монадических комбинаторов 
    парсеров, основанный на расширяемых эффектах. Исходный код библиотеки доступен в 
    репозитории на GitHub~\cite{mdParse}. 

    \item На основе метода из пункта 1 автором реализован парсер подмножества языка
    Markdown, расширенного \LaTeX-вставками, а также кодогенератор в 
    \lstinline{HTML}. Реализация, инструкция по сборке и пример использования 
    доступны в репозитории на GitHub~\cite{mdParse}. 
  \end{enumerate}

  Также в разделе 1.2 автором произведён сравнительный анализ удобства 
  практического применения двух подходов к управления вычислительными эффектами: 
  трансформеров монад и расширяемых эффектов. 

\section{Практическая значимость}

  Использование библиотеки \lstinline{monoid-subclasses}~\cite{monoidSubclassesHackage}
  при реализации комбинаторов парсеров  
  позволило повысить уровень абстракции и наделить парсеры 
  полиморфностью по входному потоку. Это качество 
  придаёт разработанной автором библиотеке \emph{практическую значимость}, 
  обеспечивая возможность работы с различными строковыми типами языка \lstinline{Haskell}.  

  Реализованная автором прототип библиотеки синтаксических анализаторов, 
  основанной на~\lstinline{extensible-effects}, имеет существенное отличие от популярных аналогов: в качестве абстракции для управления вычислительными эффектами используются расширяемые эффекты, 
  позволяющие использовать несколько однородных эффектов и не требующие ручного 
  описания экземпляров монадических классов типов для каждого нового эффекта.  


\section{Предполагаемые применения полученных результатов}

  Разработанная библиотека парсеров, полиморфных по входу, может применяться 
  при реализации парсеров языков разметки, а также как часть фронтэнда компилятора: 
  для построение синтаксического дерева языка программирования языков программирования. 

  Реализованные парсер Markdown с возможность \LaTeX-вставок и HTML-кодогенератор 
  могут быть использованы в качестве системы поддержки ведения электронного конспекта
  лекций по математике и компьютерным наукам. 

\section{Направления дальнейших исследований}

  Исследования в области поиска подходов к управлению вычислительными эффектами 
  продолжаются: появляются новые средства для типизации побочных эффектов, 
  которые нуждаются в апробации на практических задачах. 
