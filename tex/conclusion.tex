\chapter{Заключение}

\section{Основные результаты работы} 
  \begin{enumerate}
    
    \item Реализована библиотека монадических комбинаторов парсеров, основанная 
    на трансфорерах монад. Исходный код доступен в репозитории на GitHub~\cite{mdParse}. Библиотечные парсеры доступны в файле \lstinline{src/Parsers/Parsers.hs}. 
    Использование библиотеки \lstinline{monoid-subclasses}~\cite{monoidSubclassesHackage} 
    позволило наделить парсеры полиморфностью по входному потоку, это качество 
    придаёт разработанной автором библиотеке парсеров \emph{практическую значимость}, 
    обеспечивая возможность работы с различными строковыми типами языка 
    \lstinline{Haskell}.  

    \item Частично реализован прототип библиотеки монадических комбинаторов 
    парсеров, основанная на расширяемых эффектах. Исходный код доступен в 
    репозитории на GitHub~\cite{mdParse}. Реализованная автором прототип библиотеки
    имеет существенное отличие от популярных аналогов: в качестве абстракции 
    для управления вычислительными эффектами используются расширяемые эффекты.

    \item Автором произведён сравнительный анализ удобства практического применения 
    двух подходов к управления вычислительными эффектами: трансформеров монад 
    и расширяемых эффектов. 

    \item На основе библиотеки из пункта 1 автором реализован парсер подмножества языка
    Markdown, расширенного \LaTeX-вставками, а также кодогенератор в 
    \lstinline{HTML}. Реализация, инструкция по сборке и пример использования 
    доступны в репозитории на GitHub~\cite{mdParse}. 
  \end{enumerate}

\section{Предполагаемые применения полученных результатов}

  Разработанная библиотека парсеров, полиморфных по входу, может применяться 
  при реализации парсеров языков разметки, а также как часть фронтэнда компилятора: 
  для построение синтаксического дерева языка программирования языков программирования. 

  Реализованные парсер Markdown с возможность \LaTeX-вставок и HTML-кодогенератор 
  могут быть использованы в качестве системы поддержки ведения электронного конспекта
  лекций по математике и компьютерным наукам. 

\section{Направления дальнейших исследований}

  Исследования в области поиска подходов к управлению вычислительными эффектами 
  не стоят на месте: появляются новые средства для типизации побочных эффектов, 
  которые нуждаются в апробации на практических задачах. 
