\Intro

Функциональное программирование --- парадигма программирования, которая трактует
программу как вычисление значения некоторой математической функции.

Функциональный стиль построения парсеров заключается в том, что парсер
представляется функцией, которая принимает на вход строку символов и строит
по ней некоторое абстрактное синтаксическое дерево. Удобно считать, что не каждый
парсер полностью потребляет входную строку, это даёт возможность
построения результирующего парсера по частям, из примитивных. Также необходимо
иметь средства для обработки некорректного входа. Для этого существуют разные
способы: можно сообщать об ошибке разбора, либо заменять неудачу списком
успехов~\cite{wadlerSuccess}. Высказанные пожелания о характеристиках
парсера могут быть отражены следующим типом языка Haskell:

\begin{figure}[h]
\begin{lstlisting}
type Parser a = String -> [(a,String)]
\end{lstlisting}
\caption{Тип Parser}
\end{figure}

Современные исследования в области функционального программирования предоставляют
несколько способов  типизации вычислений с побочными эффектами. Важной и
актуальной задачей является апробация этих абстракций на практических
задачах. Синтаксический анализ традиционно считается одной из модельных
задач в мире функционального программирования и хорошо подходит для практического
применения новых абстракций.

\emph{Объектом исследования} работы являются технологии функционального программирования,
предназначенные для описания вычислений с побочными эффектами, производится
сравнение методов построения вычислений с несколькими побочными эффектами,
и приводятся способы их использования для построения библиотек функциональных
парсеров.

\emph{Целью работы} является анализ доступных средств типизации вычислений с побочными
эффектами и анализ их положительных и отрицательных сторон в контексте
разработки библиотек комбинаторов парсеров. В качестве практической части предлагалось 
разработать прикладную программу для трансляции исходных текстов на языке разметки 
Markdown в HTML.

В качестве отправной точки для разработки библиотеки монадических комбинаторов
парсеров используются результаты статьи~\cite{monParsing}. Описание вычислений
с несколькими побочными эффектами производится с помощью трансформеров монад
~\cite{monadTransformers}, а также расширяемых эффектов~\cite{extEffects} ---
альтернативного трансформерам монад подхода. Для построения полиморфных по
входному потоку парсеров применяются специализированные моноиды, представленные
в статье~\cite{monoids}.

Таким образом, для достижения поставленной цели необходимо реализовать экспериментальные
прототипы библиотек комбинаторов парсеров: один, основанный на
трансформерах монад, и второй, основанный на расширяемых эффектах, а также 
транслятор Markdown в HTML. Обе библиотеки должны использовать абстрактное 
моноидальное представление входного потока.
