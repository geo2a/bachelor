\Intro

\textbf{Актуальность темы.} Синтаксические анализаторы (парсеры) является необходимыми  
компонентами различных программных систем. Они используются для разбора исходного 
кода на языке программирования или разметки, с целью получения промежуточного 
представления для дальнейшей обработки данных. Парсеры могут быть сгенерированы 
автоматически с помощью специальных программ --- генераторов синтаксических 
анализаторов, или же написаны программистом в ручную. 

Синтаксические анализаторы, как и любые другие программные продукты, могут 
содержать ошибки реализации. Одним из методов повышения надёжности программных 
систем является использование для их разработки языков программирования со 
строкой статической типизацией и развитой системой типов. Современные функциональные 
языки программирования предоставляют возможность легковесной верификации программ
путём строгого контроля типов. 

Функциональное программирование --- парадигма программирования, которая трактует
программу как вычисление значения некоторой математической функции. 
Функциональный стиль построения парсеров заключается в том, что парсер
представляется функцией, которая принимает на вход строку символов и строит
по ней некоторое абстрактное синтаксическое дерево. Удобно считать, что не каждый
парсер полностью потребляет входную строку, это даёт возможность
построения результирующего парсера по частям, из примитивных. Также необходимо
иметь средства для обработки некорректного входа. Для этого существуют разные
способы: можно сообщать об ошибке разбора, либо заменять неудачу списком
успехов~\cite{wadlerSuccess}. Высказанные пожелания о характеристиках
парсера могут быть отражены следующим типом языка Haskell:

\begin{figure}[h]
\begin{lstlisting}
type Parser a = String -> [(a,String)]
\end{lstlisting}
\caption{Тип Parser}
\end{figure}

Типы вида~\lstinline{Parser a} можно рассматривать как вычисления с определённым 
побочным эффектом. Для расширения возможностей и повышения удобства синтаксического 
анализа можно расширять набор эффектов функций-парсеров.    
Современные исследования в области функционального программирования предоставляют
различные методы типизации вычислений с побочными эффектами. 
Актуальной задачей является апробация этих абстракций на практических
задачах. Синтаксический анализ традиционно считается одной из модельных
задач в мире функционального программирования и хорошо подходит для практического
применения новых абстракций.

\textbf{Объектом исследования} является теория построения синтаксических анализаторов 
с использованием статически типизированных функциональных языков программирования.  

\textbf{Предметом исследования} является методы функционального программирования, 
предназначенные для описания вычислений с несколькими побочными эффектами, 
в приложении к построению синтаксических анализаторов.  

\textbf{Целью работы} является развитие гибких и выразительных методов построения 
синтаксических анализаторов, на основе современных подходов к контролю вычислительных
эффектов.
К достижению этой цели приводит решение следующих \textbf{задач}:
\begin{enumerate}
  \item Разработка библиотеки монадических комбинаторов парсеров, основанных на 
  трансформерах монад.
  \item Разработка библиотеки монадических комбинаторов парсеров, основанных на 
  расширяемых эффектах.
  \item Разработка транслятора подмножества языка~\lstinline{Markdown} с возможностью
  \LaTeX~вставок в~\lstinline{HTML}.
\end{enumerate}  

В качестве отправной точки для разработки библиотеки монадических комбинаторов
парсеров используются результаты статьи~\cite{monParsing}. Описание вычислений
с несколькими побочными эффектами производится с помощью трансформеров монад
~\cite{monadTransformers}, а также расширяемых эффектов~\cite{extEffects} ---
альтернативного трансформерам монад подхода. Для построения полиморфных по
входному потоку парсеров применяются специализированные моноиды, представленные
в статье~\cite{monoids}.
