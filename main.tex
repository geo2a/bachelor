% В этом файле следует писать текст работы, разбивая его на
% разделы (section), подразделы (section) и, если нужно,
% главы (chapter).

% Предварительно следует указать необходимую информацию
% в файле SETUP.tex

\input{tex/preamble.tex}

\begin{document}

\begin{titlingpage}
    \begin{abstract}
      \centerline{\Large{\textbf{Аннотация}}}
      В работе разрабатываются методы построения библиотек синтаксических анализаторов 
      на функциональных языках программирования с развитыми системами типов. 
      Производится сравнение двух подходов к управлению вычислительными эффектами, 
      трансформеров монад и расширяемых эффектов, с точки зрения эффективности 
      построения библиотек монадических комбинаторов парсеров. Описываются 
      прототипные реализации библиотек парсеров, основанных на этих подходах. 
      Производится построение транслятора подмножества~\lstinline{Markdown} 
      с возможностью \LaTeX"/вставок в~\lstinline{HTML}.

      \vspace{16pt}

      \textbf{Ключевые слова}: синтаксический анализ, парсер, монадические комбинаторы 
      парсеров, функциональное программирование, вычислительные эффекты 
    \end{abstract}
\end{titlingpage}

\tableofcontents

\Intro

\textbf{Актуальность темы.} Синтаксические анализаторы (парсеры) является необходимыми  
компонентами различных программных систем. Они используются для разбора исходного 
кода на языке программирования или разметки, с целью получения промежуточного 
представления для дальнейшей обработки данных. Парсеры могут быть сгенерированы 
автоматически с помощью специальных программ --- генераторов синтаксических 
анализаторов, или же написаны программистом в ручную. 

Синтаксические анализаторы, как и любые другие программные продукты, могут 
содержать ошибки реализации. Одним из методов повышения надёжности программных 
систем является использование для их разработки языков программирования со 
строгой статической типизацией и развитой системой типов. Современные функциональные 
языки программирования предоставляют возможность легковесной верификации программ
путём строгого контроля типов. 

Функциональное программирование --- парадигма программирования, которая трактует
программу как вычисление значения некоторой математической функции. 
Функциональный стиль построения парсеров заключается в том, что парсер
представляется функцией, которая принимает на вход поток символов и строит
по нему некоторое абстрактное синтаксическое дерево. Удобно считать, что не каждый
парсер полностью потребляет входной поток, это даёт возможность
построения результирующего парсера по частям, из примитивных. Также необходимо
иметь средства для обработки некорректного входа. Для этого существуют разные
способы: можно сообщать об ошибке разбора, либо заменять неудачу списком
успехов~\cite{wadlerSuccess}. Высказанные пожелания о характеристиках
парсера могут быть отражены следующим типом языка Haskell:

\begin{figure}[h]
\begin{lstlisting}
type Parser a = String -> [(a,String)]
\end{lstlisting}
\caption{Тип Parser}
\end{figure}

Типы вида~\lstinline{Parser a} можно рассматривать как вычисления с определённым 
побочным эффектом. Для расширения возможностей и повышения удобства синтаксического 
анализа можно расширять набор эффектов функций-парсеров.    
Современные исследования в области функционального программирования предоставляют
различные методы типизации вычислений с побочными эффектами. 
Актуальной задачей является апробация этих абстракций на практических
задачах. Синтаксический анализ традиционно считается одной из модельных
задач в мире функционального программирования и хорошо подходит для практического
применения новых абстракций.

\textbf{Объектом исследования} является теория построения синтаксических анализаторов 
с использованием статически типизированных функциональных языков программирования.  

\textbf{Предметом исследования} является методы функционального программирования, 
предназначенные для описания вычислений с несколькими побочными эффектами, 
в приложении к построению синтаксических анализаторов.  

\textbf{Целью работы} является развитие гибких и выразительных методов построения 
синтаксических анализаторов, на основе современных подходов к контролю вычислительных
эффектов.
К достижению этой цели приводит решение следующих \textbf{задач}:
\begin{enumerate}
  \item Разработка метода построения монадических комбинаторов парсеров, 
  основанного на трансформерах монад.
  \item Разработка метода построения монадических комбинаторов парсеров, 
  основанного на расширяемых эффектах.
  \item Разработка транслятора подмножества языка~\lstinline{Markdown} с возможностью
  \LaTeX"/вставок в~\lstinline{HTML}.
\end{enumerate}  

В качестве отправной точки для разработки библиотеки монадических комбинаторов
парсеров используются результаты статьи~\cite{monParsing}. Описание вычислений
с несколькими побочными эффектами производится с помощью трансформеров монад
~\cite{monadTransformers}, а также расширяемых эффектов~\cite{extEffects} ---
альтернативного трансформерам монад подхода. Для повышения гибкости разрабатываемых 
методов и построения полиморфных по входному потоку парсеров применяются 
специализированные моноиды, представленные в статье~\cite{monoids}.


\chapter{Обзор подходов к управлению вычислительными эффектами и построению
монадических парсеров}

\section{Вычисления с побочными эффектами}

  Возможность статического контроля побочных эффектов является одним из принципиальных 
  преимуществ статически типизированных языков программирования с богатыми системами типов. 
  Отделение ``чистых'' функций от вычислений, обременённых взаимодействием с 
  устройствами ввода-вывода и другими побочными эффектами, позволяет с большей
  уверенностью рассуждать о надёжности разрабатываемой программы. 

  Далее будет произведён обзор и сравнение двух подходов к управлению вычислительными 
  эффектами.   

\section{Сравнение трансформеров монад и расширяемых эффектов}
 
  В данной работе главными параметрами, относительно которых производилось сравнение, 
  выступали абстрактность и гибкость, то есть: насколько общим является тот или иной подход к управлению вычислительными эффектами, а также насколько удобно его можно подогнать к 
  специфической задаче, в данном случае, задаче синтаксического анализа.  

  \subsection{Трансформеры монад}

    В статье~\cite{monadTransformers} описаны объекты, трансформеры монад,
    которые можно использовать в качестве блоков для построения типов, описывающих
    вычисления с побочными эффектами. Каждый из трансформеров монад позволяет
    добавить некоторый вычислительный эффект к внутренней монаде, при
    этом для результирующего типа также возможно построение экземпляра класса
    типов \lstinline{Monad}.

    На данный момент трансформеры монад являются распространённым способом
    построения вычислений с несколькими побочными эффектами. Однако эта
    техника имеет недостатки: проблема невозможности автоматического
    \emph{подъёма} (англ. \lstinline{lift}) при наличии в стеке двух эффектов одного
    рода, связанная с первой проблема статически определённого порядка эффектов в
    стеке, а также невозможность скомбинировать две произвольные монады.

    В листинге~\ref{listing:mtlReadersLift} приведён пример монадической функции,
    которая должна иметь два различных конфигурационных параметра. Если не
    произвести явный подъём, то компиляция этой функции завершится с ошибкой
    проверки типов вида~\ref{listing:mtlCompileError}: компилятор не в состоянии
    самостоятельно вывести нужный тип для функции \lstinline{ask}.

    \begin{figure}[t]
    \begin{lstlisting}
    adder :: ReaderT String (Reader Int) Int
    adder = do
      str <- ask
      num <- lift ask
      return $ num + read str

    runnerForAdder = runReader (runReaderT adder "2") 3
    \end{lstlisting}
    \caption{Необходимость явного подъёма во внутреннюю монаду}
    \label{listing:mtlReadersLift}
    \end{figure}

    Порядок монад в стеке определён статически и закодирован в типе функции,
    в листинге~\ref{listing:mtlDifferentReadersLift} представлена та же функция,
    эквивалентная по смыслу, но имеющая другой тип. Значительной деталью также
    является тот факт, что в реализации этой функции \lstinline{lift} применяется
    к другому вызову функции \lstinline{ask}.

    \begin{figure}[t]
    \begin{lstlisting}
    adder :: ReaderT Int (Reader String) Int
    adder = do
      str <- lift ask
      num <- ask
      return $ num + read str

    runnerForAdder = runReader (runReaderT adder 3) "2"
    \end{lstlisting}
    \caption{Функция из листинга~\ref{listing:mtlReadersLift} с другим порядком монад в стеке}
    \label{listing:mtlDifferentReadersLift}
    \end{figure}

    Функция \lstinline{runnerForAdder} из листингов~\ref{listing:mtlReadersLift}
    и~\ref{listing:mtlDifferentReadersLift}, запускающая вычисление, полностью
    определяется типом вычисления, она производит развёртку стека монад.
    В следующем подразделе будет рассмотрено средство комбинирования вычислительных
    эффектов, при использовании которого порядок на наборе эффектов устанавливается
    именно функцией, запускающей вычисление, а сам набор является неупорядоченным.

    \begin{figure}[t]
    \begin{lstlisting}
    No instance for (MonadReader Int (ReaderT String (Reader Int)))
          arising from a use of ask
    \end{lstlisting}
    \caption{Ошибка типов при отсутствии явного подъёма}
    \label{listing:mtlCompileError}
    \end{figure}

    Следует сделать замечание относительно предыдущего примера: разнородная
    конфигурационная информация могла быть объединена в алгебраический тип данных,
    и тогда нужда в использовании двух эффектов \lstinline{Reader} отпала бы,
    однако такое решение может не сработать для других эффектов, поэтому необходимо
    искать универсальный способ.

    Ещё одной проблемой трансформеров монад является необходимость описывать 
    \lstinline{n} экземпляров классов типов, 
    по количеству уже существующих трансформеров, при введении нового трансформера.  
    Предположим, что возникла необходимость ввести класс типов для некоторого нового 
    вычислительного эффекта: листинг~\ref{listing:mtlNewEffect}. 

    \begin{figure}[t]
    \begin{lstlisting}
    class Monad m => MonadNew a m where
      action1 :: m a
      action2 :: m ()
    \end{lstlisting}
    \caption{Класс типов для описания возможностей нового вычислительного эффекта}
    \label{listing:mtlNewEffect}
    \end{figure}

    Для того, чтобы полноценно использовать новый эффект, необходимо задать 
    правила ``поднятия'' (\lstinline{lift}), то есть, описать экземпляры классов типов для 
    всех трансформеров из~\lstinline{mtl}. В листинге~\ref{listing:mtlNewEffectInstances}
    приведены примеры экземпляров классов типов.

    \begin{figure}[t]
    \begin{lstlisting}
    instance MonadNew a m => MonadNew a (IdentityT m) where
      action1 = lift action1
      action2 = lift action2

    instance MonadNew a m => MonadNew a (MaybeT m) where
      action1 = lift action1
      action2 = lift action2

    ...

    \end{lstlisting}
    \caption{Экземпляры классов типов для всех возможных комбинаций нового эффекта 
    с существующими}
    \label{listing:mtlNewEffectInstances}
    \end{figure}

    Ведутся активные поиски способов комбинирования монад, которые были бы лишены
    вышеперечисленных недостатков. Одним из развивающихся направлений являются
    расширяемые эффекты (\lstinline{Extensible Effects}), которе будут обсуждаться
    в следующем разделе.

  \subsection{Расширяемые эффекты}

    Расширяемые эффекты, подробно описанные в статье~\cite{extEffects}, представляют
    собой альтернативный трансформерам монад подход к описанию вычислений с поблочными
    эффектами.

    Суть подхода заключается в аналогии между вычислительными эффектами и
    клиент-серверным взаимодействием. Код, который собирается породить некоторый
    побочный эффект: совершить ввод-вывод, бросить исключение и тому подобное должен
    отправить \emph{запрос} на обработку этого эффекта особой глобальной сущности
    --- менеджеру эффектов. Запрос описывает действие, которое необходимо
    произвести, а также функцию-продолжение (англ. \emph{continuation}) для
    возобновления работы после обработки запроса.

    В ранних разработках, относящихся к такому подходу, менеджер запросов не был
    частью программы пользователя, а являлся отдельной сущностью, подобно ядру
    операционной системы, или обработчику \lstinline{IO}-действий в
    \lstinline{Haskell}. Этот глобальный внешний авторитет имел контроль над всеми
    ресурсами (файлами, памятью и другими): он обрабатывал запрос и принимал решение:
    исполнить его и продолжить выполнение запросившего кода, либо остановить
    вычисление и вернуть результат. При таком подходе нет никакой необходимости в
    указании явного порядка при комбинировании эффектов, однако недостатком является
    негибкость внешнего интерпретатора эффектов, кроме того, эффекты никак
    не отражаются в типах.

    Разработчики библиотеки~\lstinline{Extensible Effects} модифицировали концепцию:

    \begin{itemize}
      \item
    Глобальный обработчик запросов был заменён на средство, которое является частью
    пользовательской программы, некий аналог обработчиков исключений: теперь вместо
    единого менеджера для всех эффектов существуют локальные обработчики для каждого
    типа эффектов, такой подход называется \emph{алгебраическими
    обработчиками}~\cite{effAndHandl}. Каждый такой обработчик является менеджером
    для соответствующей клиентской части программы, а также и клиентом сам по себе:
    он пересылает запросы, которые не может обработать, менеджеру верхнего уровня.
      \item
    Разработана выразительная система типов-эффектов, которая отслеживает какие
    эффекты активны в данном вычислении. Эта система поддерживает особую структуру
    данных: \emph{открытое объединение} (\lstinline{Open Union}, индексированное
    типами копроизведение функторов), содержащее неупорядоченный набор
    вычислительных эффектов. Действие каждого обработчика отражается в типе:
    происходит удаление из набора эффекта, который был обработан. Таким
    образом система типов может отследить, все ли эффекты обработаны.
      \item
    Синтаксис для работы с эффектами построен по аналогии с синтаксисом для
    трансфомеров монад. Код, написанный с использованием трансформеров монад,
    может быть переведён на расширяемые эффекты с минимальными синтаксическими
    изменениями.
    \end{itemize}

    Есть два способа обозначить принадлежность эффекта \lstinline{m} открытому
    объединению \lstinline{r}. С помощью типового ограничения
    \mbox{\lstinline{Member m r},} которое означает, что вычисления имеет
    \emph{по крайней мере} один побочный эффект \lstinline{m}. Иначе, можно явно
    указать шаблон \lstinline{m :> r'} разложения набора \lstinline{r} на
    эффект \lstinline{m} и оставшийся набор \lstinline{r'}, что аналогично
    теоретико-множественному обозначению $\{$\lstinline{m}$\}$ $\cup$
    \lstinline{r'}. Тип \lstinline{Void} играет роль $\varnothing$, то есть
    вычисление с типом \lstinline{Eff Void a} является чистым, а вычисления с
    типом \lstinline{Eff (Reader Int :> Reader Bool :> Void)} может иметь побочный
    эффект обращения к двум средам конфигурационной информации.

    Авторы также указывают, что ограничение
    \mbox{\lstinline{Member (Reader Int) r}} выглядит похоже на ограничение
    принадлежности классу типов \lstinline{MonadReader}, более того, возможно
    объявить экземпляр этого класса для монады \lstinline{Eff r}. Однако в этом нет
    необходимости, так как тип \lstinline{Eff r} является более выразительным:
    в листинге~\ref{listing:extEff2Readers} приводится пример работы с функцией,
    имеющей два эффекта \lstinline{Reader}. Эта функция демонстрирует одно из
    преимуществ расширяемых эффектов перед трансформерами монад: пропадает
    необходимость явного вызова функции \lstinline{lift}, которую можно наблюдать
    в листингах~\ref{listing:mtlReadersLift}
    и~\ref{listing:mtlDifferentReadersLift}. Каждое вхождение функции
    \lstinline{ask} обращается к своей собственной среде, определяемой типом.

    \begin{figure}[t]
    \begin{lstlisting}
    adder :: ( Member (Reader Int) r
             , Member (Reader String) r
             ) => Eff r Int
    adder = do
      num <- ask
      str <- ask
      return $ num + read str

    runAdder = run $ runReader (runReader adder "2") (1 :: Int)
    \end{lstlisting}
    \caption{Пример функции с двумя средами конфигурации.}
    \label{listing:extEff2Readers}
    \end{figure}

    Другое важное отличие расширяемых эффектов от трансформеров монад ---
    неупорядоченность набора эффектов до запуска вычисления --- никак не проявило
    себя в примере из листинга~\ref{listing:extEff2Readers}, потому что были
    использованы одинаковые эффекты. В листинге~\ref{listing:extEffOrdering}
    рассмотрен синтетический (для краткости) пример, демонстрирующий динамическую
    установку порядка на множестве эффектов. Функция из примера считает до нуля и
    завершается с исключением. Если запускать вычисление в порядке, представленном
    в \lstinline{runCountdown1}, то результатом будет \lstinline{Nothing},
    сообщающий об исключении. При запуске в порядке \lstinline{runCountdown2},
    результатом будет пара \lstinline{(0,Nothing)} из последнего состояния и
    сообщения об исключении.

    \begin{figure}[t]
    \begin{lstlisting}
    countdown :: ( Member Fail r
                 , Member (State Int) r
                 ) => Eff r ()
    countdown = do
      state <- get
      if state == (0 :: Int)
      then die
      else put (state - 1) >> countdown

    runCountdown1 n = run $ runFail $ runState (n :: Int) $ countdown

    runCountdown2 n = run $ runState (n :: Int) $ runFail $ countdown
    \end{lstlisting}
    \caption{Порядок на множестве эффектов определяется динамически}
    \label{listing:extEffOrdering}
    \end{figure}

  \subsection{Резюме}

    Оба описанных выше подхода имеют свои достоинства и недостатки. Так, концептуально, 
    расширяемые эффекты являются более прогрессивным и гибким методом контроля 
    вычислений с побочными эффектами, главным образом за счёт возможности использовать   
    несколько однородных эффектов без потери автоматического поднятия на нужный уровень, 
    а также отсутствии необходимости в явном описании огромного количества тривиальных
    экземпляров классов типов для обеспечения механизма ``поднятия''.  
    Однако, расширяемые эффекты ещё не успели получить такого широкого распространения, 
    как трансформеры монад, в следствии чего их нельзя считать готовыми к использованию 
    в критических приложениях. 

    Трансформеры монад же, в свою очередь, являются зрелой концепцией, имеющей несколько
    реализаций в виде Haskell-модулей, и использующейся в огромном количестве кода. 

\section{Монадические функциональные парсеры}

  Популярной библиотекой монадических парсеров является библиотека
  \lstinline{Parsec}~\cite{parsec}. Это библиотека промышленного уровня,
  используемая во многих проектах, например в универсальном конвертере документов
  \lstinline{Pandoc}~\cite{pandoc}. Преимуществами \lstinline{Parsec} являются
   его гибкость и подробность сообщений об ошибках.

  Другой популярной библиотекой является
  \lstinline{attoparsec}~\cite{attoparsec}. При разработке
  \lstinline{attoparsec} акцент был сделан на скорость, поэтому было принято
  решение ограничиться только одним типом \lstinline{ByteString} и пожертвовать
  информативностью сообщений об ошибках. Основным предназначением
  \lstinline{attoparsec} является анализ сетевых протоколов.

  Недостатками обеих библиотек является недостаточная абстрактность в плане типов
  данных для представления входного потока. Применение классов типов,
  представленных в предыдущем подразделе, позволит получить абстрактный код
  способный работать с любыми строковыми типами.

  Архитектура обеих библиотек использует концепцию трансформеров монад для
  представления вычислений с несколькими побочными эффектами. Необходимо
  исследовать другие концепции многоэффектных вычислений в приложении к
  монадическим парсерам.

\section{Выбранные методы}

Для реализации поставленных практических задач использовалась концепция 
трансформеров монад~\cite{monadTransformers} и основанная на ней библиотека
\lstinline{MTL}~\cite{mtlHackage}, расширяемые эффекты~\cite{extEffects} 
и библиотека \lstinline{extensible-effects}~\cite{extensibleEffectsHackage} а также библиотека
\lstinline{monoid-subclasses}~\cite{monoidSubclassesHackage}.


\chapter{Теоретические основы функционального программирования и элементы теории 
         категорий}

Функциональное программирование --- парадигма программирования, которая трактует 
программу как вычисление значения некоторой математической функции.

Корни функционального программирования уходят в $\lambda$"/исчисление,
формальную систему, разработанную в 1930-х годах для решения
Entscheidungsproblem~\cite{entscheidungsproblem}.

Одним из главных преимуществ функциональных языков программирования считается
высокий уровень абстракции, выразительность и высокий коэффициент повторного
использования кода. Для достижения этих свойств в языках реализуются такие
средства как параметрический полиморфизм, функции высших порядков, каррирование,
алгебраические типы данных, классы типов и другие. Использование неизменяемых
данных позволяет существенно упростить отладку программ.

\section{Классы типов}

  Классы типов позволяют накладывать ограничения на типы, определяя некоторый
  набор операций, которые могут производиться над типами, принадлежащими к
  некоторому классу. Иными словами, класс типов определяет интерфейс,
  через который можно взаимодействовать с типом.

  Рассмотрим определение стандартного класса типов языка \lstinline{Haskell},
  отвечающего за возможность сравнения на равенство:

  \begin{figure}[h]
  \begin{lstlisting}
  class Eq a where
    (==) :: a -> a -> Bool
    (/=) :: a -> a -> Bool
  \end{lstlisting}
  \caption{Класс типов, для которых введено отношение эквиваленции}
  \label{listing:Eq}
  \end{figure}

  Чтобы воспользоваться возможностями класса типов для какого-то конкретного типа
  необходимо объявить этот тип экземпляром класса типов, во многих случаях это
  может быть сделано неявно, благодаря автоматическому порождению экземпляров
  компилятором \lstinline{GHC}.

  Возможность автоматического порождения экземпляра нужного класса типов для
  алгебраического типа данных:
  \begin{figure}[h]
  \begin{lstlisting}
  data Numbers = One | Two | Three deriving Eq

  ghci> One /= Two
  True
  \end{lstlisting}
  \caption{Автоматическое порождение экземпляров классов типов}
  \label{listing:List}
  \end{figure}


  В некоторых случаях необходимо описать экземпляр явно, такая возможность тоже
  существует.

  \begin{figure}[h]
  \begin{lstlisting}
  data Foo = Foo {x :: Integer, str :: String}

  instance Eq Foo where
    (Foo x1 str1) == (Foo x2 str2) = (x1 == x2) &&
                                     (str1 == str2)
  \end{lstlisting}
  \caption{Явное описание экземпляра класса типов}
  \label{listing:Instance}
  \end{figure}

  Как уже говорилось ранее, код, написанный на языке программирования
  \lstinline{Haskell} имеет очень высокий уровень абстракции. Далее будут
  рассмотрены два классы типов, представляющие структуры, введённые в язык
  \lstinline{Haskell} под влиянием исследований в области связи теоретической
  информатики и теории категорий. Эти структуры представляют абстракцию для
  вычислений с побочными эффектами. В дальнейшем изложении эти классы типов будут
  использоваться для построения парсеров.

\section{Функтор}

  Рассмотрим определение класса типов \lstinline{Functor} из стандартной
  библиотеки \lstinline{Haskell}:

  \begin{figure}[h]
  \begin{lstlisting}
    class Functor f where
      fmap :: (a -> b) -> f a -> f b
  \end{lstlisting}
  \caption{Класс типов \lstinline{Functor}}
  \label{listing:Functor}
  \end{figure}

  Как видно из определения, каждый тип, являющийся функтором, должен предоставлять
  функцию \lstinline{fmap}.

  Сформировались два основных неформальных описания функтора: контейнер,
  содержащий в себе значения определённого типа и вычисление, которое производится
  в некотором контексте. Согласно первой из этих трактовок, функция
  \lstinline{fmap} применяет подаваемую ей на вход функцию к значениям в
  контейнере и возвращает изменённые значения, с сохранением структуры контейнера.
  Если же говорить в терминах вычислительных контекстов, то \lstinline{fmap}
  модифицирует вычисление в контексте, но сам контекст остаётся неизменным.

  Упомянутые выше списки языка \lstinline{Haskell} являются функторами, для них в
  качестве \lstinline{fmap} можно выбрать функцию \lstinline{map}.

  Важно заметить, что не любой тип, для которого определена функция
  \lstinline{fmap} является функтором, необходимо также потребовать выполнения
  двух уравнений, называемых законами функтора:

  \begin{figure}[h]
  \begin{lstlisting}
  fmap id = id
  fmap (g . h) = (fmap g) . (fmap h)
  \end{lstlisting}
  \caption{Законы функтора}
  \label{listing:FunctorLaws}
  \end{figure}

  Здесь \lstinline{id} --- тождественная функция, а \lstinline{.} --- инфиксный
  оператор композиции функций. Эти уравнения отражают теоретико-категорное
  определение функтора как отображения между категориями, сохраняющее единичный
  морфизм и композицию. В случае \lstinline{Haskell} рассматриваются эндофункторы
  над категорией \lstinline{Hask} --- категорией типов языка \lstinline{Haskell}.

\section{Аппликативный функтор}

  Аппликативный функтор является специализацией функтора, допускающей применение
  функции, находящейся внутри некоторого контекста, к значению в таком же
  контексте. Аппликативные функторы также иногда называют \emph{идиомами}
  (англ. idioms).

  В языке \lstinline{Haskell} для представления аппликативных функторов
  используется стандартный класс типов \lstinline{Applicative},
  описанный в модуле \lstinline{Control.Applicative}.

  \begin{figure}[h]
  \begin{lstlisting}
  class (Functor f) => Applicative f where
      pure :: a -> f a
      (<*>) :: f (a -> b) -> f a -> f b
  \end{lstlisting}
  \caption{Класс типов \lstinline{Applicative}}
  \label{listing:Applicative}
  \end{figure}

  Для всякого типа, который является аппликативным функтором, должны выполняться
  законы, представленные в листинге~\ref{listing:ApplicativeLaws}.

  \begin{figure}[h]
  \begin{lstlisting}
  pure id <*> v = v

  pure (.) <*> u <*> v <*> w = u <*> (v <*> w)

  pure f <*> pure x = pure (f x)

  u <*> pure y = pure ($ y) <*> u
  \end{lstlisting}
  \caption{Законы аппликативного функтора}
  \label{listing:ApplicativeLaws}
  \end{figure}

  Любая монада является аппликативным функтором, но не каждый аппликативный
  функтор является монадой. То есть, интерфейсу класса типов
  \lstinline{Applicative} удовлетворяет больше типов, чем интерфейсу класса типов
  \lstinline{Monad}.

  С понятием аппликативного функтора связан особый синтаксис описания вычислений с
  эффектами на \lstinline{Haskell}, называемый \emph{аппликативным стилем}. Этот
  стиль имеет преимущество относительно \lstinline{do}-нотации для монад ---
  компилятор имеет возможность генерировать для него более оптимальный код.
  Многие монадические функции могут быть переписаны в аппликативном стиле.
  В листинге~\ref{listing:MonadApplicative} представлена короткая монадическая
  функция и эквивалентная ей аппликативная, задача которых прочитать из потока
  стандартного ввода две строки выполнить их конкатенацию и вернуть результат.

  \begin{figure}
  \begin{lstlisting}
  action :: IO String
  action = do
    a <- getLine
    b <- getLine
    return $ a ++ b

  action :: IO String
  action = (++) <$> getLine <*> getLine
  \end{lstlisting}
  \caption{\lstinline{do}-нотация и аппликативный стиль}
  \label{listing:MonadApplicative}
  \end{figure}

\section{Монада}

  Вершиной иерархии типов, описывающих вычисления с побочными эффектами,
  являются монады. Возникнув как средство для введения в чистый функциональный
  язык возможности выполнять операции ввода-вывода, монады были обобщены и на
  другие вычислительные эффекты, для которых есть необходимость в построении
  композиции функций, и теперь являются наиболее известной абстракцией такого
  сорта.

  Для представления монад в языке \lstinline{Haskell} используется класс типов
  \lstinline{Monad}. Кроме того, существует три закона монады, которые должны
  выполняться для каждого типа, имеющего экземпляр класса типов \lstinline{Monad}.
  Эти законы отражают тот факт, что монада является моноидом в категории
  эндофункторов. Важно знать, что система типов не гарантирует выполнения
  монадических законов, ответственность за них целиком лежит на программисте,
  который разрабатывает экземпляр \lstinline{Monad} для некоторого типа.

  \begin{figure}[h]
  \begin{lstlisting}
  class Monad m where
    (>>=) :: m a -> (a -> m b) -> m b
    (>>) :: m a -> m b -> m b
    return :: a -> m a
  \end{lstlisting}
  \caption{Класс типов \lstinline{Monad}}
  \label{listing:Monad}
  \end{figure}

  \begin{figure}[h]
  \begin{lstlisting}
  return a >>= k = k a
  m >>= return = m
  m >>= (\x -> k x >>= h) = (m >>= k) >>= h
  \end{lstlisting}
  \caption{Законы монады}
  \label{listing:MonadLaws}
  \end{figure}

  Монада естественным образом возникает при построении функциональных парсеров,
  которые описываются в третьей главе.

\section{Копроизведение}

  Копроизведением объектов $A$ и $B$ в категории называется объект $A + B$ и две
  стрелки $i_1 : A \to A + B$ и $i_2 : B \to A + B$ такие, что для всех стрелок
  $f : A \to C$ и $g : B \to C$ существует единственная стрелка
  $\langle f|g\rangle  : A + B \to C$ для которой следующая диаграмма
  коммутативна~\cite{TeorCat}.

  \begin{center}
  \begin{tikzcd}
  A \arrow{r}{f} \arrow{rd}[swap]{i_1}
  &C
  &B \arrow{ld}{i_2} \arrow{l}[swap]{g}\\
  &A+B \arrow{u}[description]{\langle f|g\rangle}
  \end{tikzcd}
  \end{center}

  Стрелки $i_1$ и $i_2$ называются каноническими вложениями.

  В случае категории множеств копроизведением является дизъюнктивное
  или размеченное объединение.

  Понятие копроизведения, в случае категории типов языка \lstinline{Haskell},
  потребуется в главе 2, для описания расширяемых эффектов.

\section{Моноиды в функциональном программировании}

  Моноидом является множество $M$ с заданной на нём бинарной ассоциативной
  операцией $*$, и в котором существует такой элемент
  $e$, что $\forall x \in M~e*x = x*e = x$. Элемент $e$ называется единицей.

  В языке \lstinline{Haskell} для представления моноидов существует специальный
  класс типов.

  \begin{figure}[h]
  \begin{lstlisting}
  class Monoid a where
    mempty  :: a
    mappend :: a -> a -> a
  \end{lstlisting}
  \caption{Моноид в \lstinline{Haskell}}
  \label{listing:Monoid}
  \end{figure}

  Cтроки с операцией конкатенации и пустой строкой в качестве единицы являются
  моноидом. С помощью класса типов \lstinline{Monoid} возможна абстрактная работа
  с любыми строковыми типами. Однако существуют ситуации, в которых интерфейса
  класса типов \lstinline{Monoid} оказывается недостаточно, возникает
  необходимость в функции, отделяющей префикс, аналогичной функции
  \lstinline{head} для списков. Для таких целей создана библиотека
  \lstinline{monoid-subclasses}~\cite{monoids}, предоставляющая
  классы типов, являющиеся моноидами специального вида, которые допускают
  необходимые дополнительные операции. В случае парсеров такой дополнительной
  операцией является аналог операции разделения списка на голову и хвост.

  \begin{figure}[h]
  \begin{lstlisting}
  splitCharacterPrefix :: t -> Maybe (Char, t)

  uncons :: [a] -> Maybe (a, [a])
  \end{lstlisting}
  \caption{Отделение атомарного префикса и взятие головы списка}
  \label{listing:Monoid}
  \end{figure}

\chapter{Методы построения библиотек синтаксических анализаторов}

Вернёмся к простейшему типу для представления синтаксического анализатора 
(листинг~\ref{listing:basicParserType}). Парсер --- это функция, 
принимающая на вход строку символов и возвращающая список пар возможных вариантов  
разбора и неразобранных остатков входного потока. Пустой список результатов 
соответствует полностью неудачному разбору, а множество результатов --- 
неоднозначности.

\begin{figure}[h]
\begin{lstlisting}
type Parser a = String -> [(a,String)]
\end{lstlisting}
\caption{Тип Parser}
\label{listing:basicParserType}
\end{figure}

Типы вида Parser a можно рассматривать как вычисления с
определённым побочным эффектом. В данном случае эксплуатируется эффект 
недетерминизма вычислений, для выражения возможности неоднозначного разбора.
Для представления эффектов такого рода в языке~\lstinline{Haskell} используется
концепция монады~\cite{wadlerMonads}. В статье~\cite{monParsing} можно найти 
подробную информацию о свойствах таких парсеров, а также описание необходимых 
экземпляров классов типов.

Для расширения возможностей и повышения удобства синтаксического анализа можно расширять
набор эффектов функций-парсеров. Далее будут представлены методы построения библиотек
монадических комбинаторов парсеров с использованием двух подходов к контролю
вычислительных эффектов: трансформеров монад и расширяемых эффектов. 

\section{Синтаксические анализаторы на основе стека трансформеров монад}

  Трансформеры (преобразователи) монад --- концепция, позволяющая придать заданной 
  монаде свойства другой монады. Возможно многократное повторение этого преобразования, 
  порождающее стек монад, которые комбинирует вычислительных эффекты, привносимые 
  каждой отдельной монадой.

  В статье~\cite{monParsing} предлагается способ разложения типа из 
  листинга~\ref{listing:basicParserType} на стек из двух монад: монады для вычислений 
  с состоянием и списковой монады, обеспечивающей эффект недетерминизма. Таким образом, 
  тип приобретает вид, представленный в листинге~\ref{listing:monadStackParserType}.

  \begin{figure}[h]
  \begin{lstlisting}
  type Parser a = StateT String [] a
  \end{lstlisting}
  \caption{Парсер как простой стек монад}
  \label{listing:monadStackParserType}
  \end{figure}

  Концепция трансформеров монад позволяет добавлять к стеку различные монады, 
  предоставляющие различные полезные свойства: вычисления в конфигурируемом окружении, 
  потенциально ошибочные вычисления, ведение лога и другие. В разработанной автором 
  библиотеке монадических комбинаторов парсеров~\cite{mdParse} используется многоуровневый 
  стек монад, представленный в листинге~\ref{listing:hugeMonadStackParserType}. Это 
  представления парсера также является параметризованным типом входного потока.
  Типы~\lstinline{ParserState} и~\lstinline{ErrorReport} представляют собой 
  алгебраические типы данных для представления состояния парсера и возможной 
  ошибки разбора соответственно.

  \begin{figure}[h]
  \begin{lstlisting}
  newtype Parser t a = Parser (  
      StateT (ParserState t) (Either (ErrorReport t)) a
    ) deriving ( Functor, Applicative, Monad
               , MonadState (ParserState t)
               , MonadError (ErrorReport t)
               )
  \end{lstlisting}
  \caption{Парсер как стек монад}
  \label{listing:hugeMonadStackParserType}
  \end{figure}

  Самым низкоуровневым примитивом, на основе которого строятся все последующие 
  комбинаторы, является парсер, потребляющий из входного потока единственный элемент
  (листинг~\ref{listing:mtlParsersItem}).

  \begin{figure}[h]
  \begin{lstlisting}
  item :: TM.TextualMonoid t => Parser t Char
  item = do
    state  <- get
    let s = TM.splitCharacterPrefix . remainder $ state
    case s of 
      Nothing -> throwError (EmptyRemainder "item",state)
      Just (c,rest) -> do  
        let (c,rest) = fromJust s
        put (ParserState {position = updatePos (position state) c, remainder = rest})
        return c  
  \end{lstlisting}
  \caption{Потребление одного элемента входного потока парсером на основе трансформеров 
  монад}
  \label{listing:mtlParsersItem}
  \end{figure} 

  В листингах~\ref{listing:mtlParserSat} и~\ref{listing:mtlParserString} приведены ещё дав примера парсеров из разработанной библиотеки: 
  распознавание символа, удовлетворяющего предикату и заданной строки символов.

  \begin{figure}[h]
  \begin{lstlisting}
  sat :: TM.TextualMonoid t => (Char -> Bool) -> Parser t Char
  sat p = do
    state <- get
    x <- item `overrideError` (EmptyRemainder "sat")
    if p x then return x else 
      throwError (UnsatisfiedPredicate "general",state)
  \end{lstlisting}
  \caption{}
  \label{listing:mtlParserSat}
  \end{figure}

  \begin{figure}[h]
  \begin{lstlisting}
  string :: TM.TextualMonoid t => String -> Parser t String
  string s = do
    state <- get
    (mapM char s) `overrideError` 
      (UnsatisfiedPredicate ("string " ++ s))
  \end{lstlisting}
  \caption{}
  \label{listing:mtlParserString}
  \end{figure}

  Для осуществления разбора необходимо описать функцию, которая ``запускает''
  вычисление (листинг~\ref{listing:mtlParseRun}). При использовании трансформеров 
  монад порядок обработки вычислительных эффектов статически зафиксирован в 
  типе стека монад. 
  
  \begin{figure}[h]
  \begin{lstlisting}
parse :: TM.TextualMonoid t => 
  Parser t a -> t -> Either (ErrorReport t) (a,ParserState t)
parse (Parser p) s = 
  runStateT p (ParserState {remainder = s, position = initPos})
    where initPos = (1,1)
  \end{lstlisting}
  \caption{}
  \label{listing:mtlParseRun}
  \end{figure}


  Следует сказать, что концепция трансформеров монад имеет определённое удобство 
  для программирования благодаря своей зрелости и популярности. Однако, как уже
  обсуждалось ранее в разделе 1, этот подход обладает довольно низкой гибкостью, 
  не позволяет добавлять к стеку несколько однородных эффектов (например, два 
  трансформера~\lstinline{StateT}) без потери возможности автоматического 
  подъёма (~\lstinline{lift}) и требует описания большого числа экземпляров классов 
  типов. 

  Рассмотрим далее методы построения монадических комбинаторов парсеров, основанные
  на расширяемых эффектах --- альтернативному трансформерам монад подходу к управлению
  вычислительными эффектами.  

\section{Синтаксические анализаторы на основе расширяемых эффектов}

  Расширяемые эффекты, подробно описанные в статье~\cite{extEffects}, представляют
  собой альтернативный трансформерам монад подход к описанию вычислений с поблочными
  эффектами.

  Суть подхода заключается в аналогии между вычислительными эффектами и
  клиент-серверным взаимодействием. Код, который собирается породить некоторый
  побочный эффект: совершить ввод-вывод, бросить исключение и тому подобное должен
  отправить \emph{запрос} на обработку этого эффекта особой глобальной сущности
  --- менеджеру эффектов. Запрос описывает действие, которое необходимо
  произвести, а также функцию-продолжение (англ. \emph{continuation}) для
  возобновления работы после обработки запроса.

  При разработке прототипа библиотеки парсеров~\cite{extEffParsers}, основанной на расширяемых 
  эффектах, автором было принято следующее архитектурное решение: 
  не создавать специальный тип для парсера, а указывать необходимый набор эффектов 
  в ограничениях каждого отдельного комбинатора.

  В листинге~\ref{listing:extEffItem} приведён базовый примитив библиотеки, 
  функция --- потребляющая единственный символ из входного потока. 
  В типовой аннотации декларированы
  производимые функцией побочные эффекты: возможная неудача при разборе и 
  вычисление с состоянием. Рассмотрим эту типовую аннотацию подробнее. 
  Ограничение~\lstinline{Member Fail r} указывает, что набор 
  эффектов~\lstinline{r} должен включать в себя эффект~\lstinline{Fail}, а 
  результирующий тип функции~\lstinline{Eff r Char} означает, что данный 
  комбинатор является вычислением с побочными эффектами из набора~\lstinline{r},
  и порождает в качестве результата значение типа~\lstinline{Char}. 

  \begin{figure}[h]
  \begin{lstlisting}
item :: ( Member Fail r
        , Member (State String) r
        ) => Eff r Char
item = do
  s <- get
  case s of 
    [] -> die
    (x:xs) -> put xs >> return x 
  \end{lstlisting}
  \caption{Потребление одного элемента входного потока парсером на основе 
  расширяемых эффектов}
  \label{listing:extEffItem}
  \end{figure} 

  В целом, с точки зрения синтаксиса языка~\lstinlie{Haskell}, описание 
  комбинаторов с помощью расширяемых эффектов не сильно отличается от 
  подхода с трансформераи монад. Это достигается за счёт того, что 
  тип~\lstinline{Eff r a} является свободной монадой, построенной по открытому 
  объединения эффектов~\lstinline{r}, которое является функтором. Благодаря
  наличию экземпляра класса типов~\lstinline{Monad} для типа~\lstinline{Eff r a}
  возможно использование do-нотации и аппликативного стиля.

  \begin{figure}[h]
  \begin{lstlisting}
sat :: ( Member Fail r
       , Member (State String) r
       ) => (Char -> Bool) -> Eff r Char
sat p = do
  (s :: String) <- get
  x <- item
  if p x then return x else (put s >> die)
  \end{lstlisting}
  \label{listing:extEffSat}
  \end{figure} 

  Расширяемые эффекты, в отличие от трансформеров монад, позволяют определять 
  порядок на множестве эффектов при запуске вычисления, что позволяет, для 
  одного и того же вычисления, получить немного разное поведение, в зависимости от
  порядка обработки эффектов этого вычисления. В листинге\ref{listing:extEffparse}
  приведены две функции для запуска синтаксического анализа. 

  \begin{figure}[h]
  \begin{lstlisting}
parse p inp = run . runState inp . runFail $ p

parse' p inp = run . runFail . runState inp $ p 
  \end{lstlisting}
  \caption{Функция для запуска вычисления}
  \label{listing:extEffparse}
  \end{figure}



\input{tex/chapter4.tex}

\chapter{Заключение}

\section{Основные результаты работы} 
  
  Автором были получены следующие \textbf{основные результаты}:

  \begin{enumerate}
    
    \item Разработан метод построения монадических комбинаторов парсеров, основанный 
    на трансфорерах монад. Исходный код библиотеки доступен в репозитории на GitHub~\cite{mdParse}. Библиотечные парсеры доступны в файле \lstinline{src/Parsers/Parsers.hs}. 

    \item Разработан прототип метода построения монадических комбинаторов 
    парсеров, основанный на расширяемых эффектах. Исходный код библиотеки доступен в 
    репозитории на GitHub~\cite{mdParse}. 

    \item На основе метода из пункта 1 автором реализован парсер подмножества языка
    Markdown, расширенного \LaTeX-вставками, а также кодогенератор в 
    \lstinline{HTML}. Реализация, инструкция по сборке и пример использования 
    доступны в репозитории на GitHub~\cite{mdParse}. 
  \end{enumerate}

  Также в разделе 1.2 автором произведён сравнительный анализ удобства 
  практического применения двух подходов к управления вычислительными эффектами: 
  трансформеров монад и расширяемых эффектов. 

\section{Практическая значимость}

  Использование библиотеки \lstinline{monoid-subclasses}~\cite{monoidSubclassesHackage}
  при реализации комбинаторов парсеров  
  позволило повысить уровень абстракции и наделить парсеры 
  полиморфностью по входному потоку. Это качество 
  придаёт разработанной автором библиотеке \emph{практическую значимость}, 
  обеспечивая возможность работы с различными строковыми типами языка \lstinline{Haskell}.  

  Реализованная автором прототип библиотеки синтаксических анализаторов, 
  основанной на~\lstinline{extensible-effects}, имеет существенное отличие от популярных аналогов: в качестве абстракции для управления вычислительными эффектами используются расширяемые эффекты, 
  позволяющие использовать несколько однородных эффектов и не требующие ручного 
  описания экземпляров монадических классов типов для каждого нового эффекта.  


\section{Предполагаемые применения полученных результатов}

  Разработанная библиотека парсеров, полиморфных по входу, может применяться 
  при реализации парсеров языков разметки, а также как часть фронтэнда компилятора: 
  для построение синтаксического дерева языка программирования языков программирования. 

  Реализованные парсер Markdown с возможность \LaTeX-вставок и HTML-кодогенератор 
  могут быть использованы в качестве системы поддержки ведения электронного конспекта
  лекций по математике и компьютерным наукам. 

\section{Направления дальнейших исследований}

  Исследования в области поиска подходов к управлению вычислительными эффектами 
  продолжаются: появляются новые средства для типизации побочных эффектов, 
  которые нуждаются в апробации на практических задачах. 


% Печать списка литературы (библиографии)
\printbibliography[heading=bibintoc%
    %,title=Библиография % если хочется это слово
]
% Файл со списком литературы: biblio.bib
% Подробно по оформлению библиографии:
% см. документацию к пакету biblatex-gost
% http://ctan.mirrorcatalogs.com/macros/latex/exptl/biblatex-contrib/biblatex-gost/doc/biblatex-gost.pdf
% и огромное количество примеров там же:
% http://mirror.macomnet.net/pub/CTAN/macros/latex/contrib/biblatex-contrib/biblatex-gost/doc/biblatex-gost-examples.pdf

\end{document}
