% В этом файле следует писать текст работы, разбивая его на
% разделы (section), подразделы (subsection) и, если нужно,
% главы (chapter).

% Предварительно следует указать необходимую информацию
% в файле SETUP.tex

\input{preamble.tex}

% Выделение красным для TODO в pdf 
\newcommand\todo[1]{\textcolor{red}{#1}} 

 % Временный сепоратор для разделов введения, должен выглядеть устрашающе
\newcommand\sep{\rule{4cm}{0.4pt}}

\begin{document}

\Intro
В данной работе рассматривается задача синтаксического анализа
файлов в формате Markdown с применением технологий функционального
программирования. В качестве языка реализации используется Haskell.
Синтаксический анализ текстов, таких как, например, исходные коды на языках программирования --- задача,поставленная ещё на самых ранних этапах развития информатики. Традиционные методы решения этой задачи, описанные, например, в книге~\autocite{DragonBook2}, используют императивный подход. 

\section{Особенности императивного подхода}

В этой работе используются методы функционального подхода к программированию, одними из основных достоинств которого является высокий уровень абстракции и большая информативность кода.

\section{Использованные в работе технологии фукционального программирования}

Ключевой концепцией функционального программирования являются так называемые функции высших порядков (higher-order function). Они представляют собой функции, в качестве параметров которых могут выступать другие функции. 

\begin{description}
  \item[Пример] 
  Рассмотрим одну из функций стандартной библиотеки языка Haskell: 
  \begin{lstlisting}
    map :: (a -> b) -> [a] -> [b]
  \end{lstlisting}
  Как видно из типовой аннотации, она принимает на вход функцию, преобразующую значение типа a к значению типа b, и список значений типа b, возвращаемым значением является список результатов применения функции к значениям из входного списка.   
\end{description}

\chapter{Классический подход к построению монады Parser}

\chapter{Построение монады Parser как комбинации более простых монад}

\chapter{Реализация парсера языка Markdown} 

% Печать списка литературы (библиографии)
\printbibliography[heading=bibintoc%
    %,title=Библиография % если хочется это слово
]
% Файл со списком литературы: biblio.bib
% Подробно по оформлению библиографии:
% см. документацию к пакету biblatex-gost
% http://ctan.mirrorcatalogs.com/macros/latex/exptl/biblatex-contrib/biblatex-gost/doc/biblatex-gost.pdf
% и огромное количество примеров там же:
% http://mirror.macomnet.net/pub/CTAN/macros/latex/contrib/biblatex-contrib/biblatex-gost/doc/biblatex-gost-examples.pdf

\end{document}
