% В этом файле следует писать текст работы, разбивая его на
% разделы (section), подразделы (section) и, если нужно,
% главы (chapter).

% Предварительно следует указать необходимую информацию
% в файле SETUP.tex

%% В этот файл не предполагается вносить изменения

% В этом файле следует указать информацию о себе
% и выполняемой работе.

\documentclass [fontsize=14pt, paper=a4, pagesize, DIV=calc]%
{scrreprt}
% ВНИМАНИЕ! Для использования глав поменять
% scrartcl на scrreprt

% Здесь ничего не менять
\usepackage [T2A] {fontenc}   % Кириллица в PDF файле
\usepackage [utf8] {inputenc} % Кодировка текста: utf-8
\usepackage [russian] {babel} % Переносы, лигатуры

%%%%%%%%%%%%%%%%%%%%%%%%%%%%%%%%%%%%%%%%%%%%%%%%%%%%%%%%%%%%%%%%%%%%%%%%
% Создание макроса управления элементами, специфичными
% для вида работы (курс., бак., маг.)
% Здесь ничего не менять:
\usepackage{ifthen}
\newcounter{worktype}
\newcommand{\typeOfWork}[1]
{
	\setcounter{worktype}{#1}
}

% ВНИМАНИЕ!
% Укажите тип работы: 0 - курсовая, 1 - бак., 2 - маг.,
% 3 - бакалаврская с главами.
\typeOfWork{3}
% Считается, что курсовая и бак. бьются на разделы (section) и
% подразделы (subsection), а маг. — на главы (chapter), разделы и
%  подразделы. Если хочется,
% чтобы бак. была с главами (например, если она большая),
% надо выбрать опцию 3.

% Если при выборе 2 или 3 вы забудете поменять класс
% документа на scrreprt (см. выше, в самом начале),
% то получите ошибку:
% ./aux/appearance.tex:52: Package scrbase Error: unknown option ` chapterprefix=

%%%%%%%%%%%%%%%%%%%%%%%%%%%%%%%%%%%%%%%%%%%%%%%%%%%%%%%%%%%%%%%%%%%%%%%%
% Информация об авторе и работе для титульной страницы

\usepackage {titling}

% Имя автора в именительном падеже (для маг.)
\newcommand {\me}{%
Г.\,А.~Лукьянов
}

% Имя автора в родительном падеже (для курсовой и бак.)
\newcommand {\byme}{%
Г.\,А.~Лукьянова%
}

% Любимый научный руководитель
\newcommand{\supervisor}%
{асс.~каф.~ИВЭ~А.\,М.~Пеленицын
}

% идентифицируем пол (только для курсовой и бак.)
\newcommand{\bystudent}{
Cтудента %студентки % Для курсовой: с большой буквы
}

% Год публикации
\date{2015}

% Название работы
\title{Функциональный парсер легковесного языка разметки Markdown
на основе комбинирования монад и моноидального представления исходного текста}

% Кафедра
%
\newboolean{needchair}
\setboolean{needchair}{true} % на ФИИТ не пишется (false), на ПМИ есть (true)

\newcommand {\thechair} {
Кафедра информатики и вычислительного эксперимента
}

\newcommand {\direction} {
Направление подготовки\\
Прикладная математика и информатика
}

%%%%%%%%%%%%%%%%%%%%%%%%%%%%%%%%%%%%%%%%%%%%%%%%%%%%%%%%%%%%%%%%%%%%%%%%
% Другие настраиваемые элементы текста

% Листинги с исходным кодом программ: укажите язык программирования
\usepackage{listings}
\lstset{
    language=Haskell,%  Язык указать здесь
    basicstyle=\small\ttfamily,
    breaklines=true,%
    showstringspaces=false,%
    frame=single,
%    xleftmargin=30pt,
%    xrightmargin=30pt
    %inputencoding=utf8x%
}
% полный список языков, поддерживаемых данным пакетом, есть,
% например, здесь (стр. 13):
% ftp://ftp.tex.ac.uk/tex-archive/macros/latex/contrib/listings/listings.pdf

% Гиперссылки: настройте внешний вид ссылок
\usepackage%
[pdftex,unicode,pdfborder=0,draft=false,%backref=page,
    hidelinks, % убрать, если хочется видеть ссылки: это
               % удобно в PDF файле, но не должно появиться на печати
    bookmarks=true,bookmarksnumbered=false,bookmarksopen=false]%
{hyperref}



\usepackage {amsmath}      % Больше математики
\usepackage {amssymb}
\usepackage {textcase}     % Преобразование к верхнему регистру
\usepackage {indentfirst}  % Красная строка первого абзаца в разделе
\usepackage {fancyvrb}     % Листинги: определяем своё окружение Verb
\usepackage {tikz-cd}       % Коммутативные диаграммы
\DefineVerbatimEnvironment% с уменьшенным шрифтом
	{Verb}{Verbatim}
	{fontsize=\small}
\usepackage{titling}


% Вставка рисунков
\usepackage {graphicx}

% Общее оформление
% ----------------------------------------------------------------
% Настройка внешнего вида

%%% Шрифты

% если закомментировать всё — консервативная гарнитура Computer Modern
\usepackage{paratype} % профессиональные свободные шрифты
%\usepackage {droid}  % неплохие свободные шрифты от Google
%\usepackage{mathptmx}
%\usepackage {mmasym}
%\usepackage {psfonts}
%\usepackage{lmodern}
%var1: lh additions for bold concrete fonts
%\usepackage{lh-t2axccr}
%var2: the package below could be covered with fd-files
%\usepackage{lh-t2accr}
%\usepackage {pscyr}

% Геометрия текста

\usepackage{setspace}       % Межстрочный интервал
\onehalfspacing

\newlength\MyIndent
\setlength\MyIndent{1.25cm}
\setlength{\parindent}{\MyIndent} % Абзацный отступ
\frenchspacing            % Отключение лишних отступов после точек
\KOMAoptions{%
    DIV=calc,         % Пересчёт геометрии
    numbers=endperiod % точки после номеров разделов
}

                            % Консервативный вариант:
%\usepackage                % ручное задание геометрии
%[%                         % (не рекомендуется в проф. типографии)
%  margin = 2.5cm,
  %includefoot,
  %footskip = 1cm
%] %
%  {geometry}

%%% Заголовки



\ifthenelse{\equal{\theworktype}{2}}{%
\KOMAoptions{%
    numbers=endperiod,% точки после номеров разделов
    headings=normal,   % размеры заголовков поменьше стандартных
    chapterprefix=true,% Печатать слово Глава в магистерской
    appendixprefix=true% Печатать слово Приложение
}
}

% шрифт для оформления глав и названия содержания
\newcommand{\SuperFont}{\Large\sffamily\bfseries}

% Заголовок главы
\ifthenelse{\value{worktype} > 1}{%
\renewcommand{\SuperFont}{\Large\normalfont\sffamily}
\newcommand{\CentSuperFont}{\centering\SuperFont}
\usepackage{fncychap}
\ChNameVar{\SuperFont}
\ChNumVar{\CentSuperFont}
\ChTitleVar{\CentSuperFont}
\ChNameUpperCase
\ChTitleUpperCase
}

% Заголовок (под)раздела с абзацного отступа
\addtokomafont{sectioning}{\hspace{\MyIndent}}

% Переделка окружения figure под листинги
\renewcommand*{\captionformat}{~---~}
\renewcommand*{\figureformat}{Листинг~\thefigure}

%%% Оглавление
\usepackage{tocloft}

% шрифт и положение заголовка
\ifthenelse{\value{worktype} > 1}{%
\renewcommand{\cfttoctitlefont}{\hfil\SuperFont\MakeUppercase}
}{
\renewcommand{\cfttoctitlefont}{\hfil\SuperFont}
}

% слово Глава
\usepackage{calc}
\ifthenelse{\value{worktype} > 1}{%
\renewcommand{\cftchappresnum}{Глава }
\addtolength{\cftchapnumwidth}{\widthof{Глава }}
}

% Очищаем оформление названий старших элементов в оглавлении
\ifthenelse{\value{worktype} > 1}{%
\renewcommand{\cftchapfont}{}
\renewcommand{\cftchappagefont}{}
}{
\renewcommand{\cftsecfont}{}
\renewcommand{\cftsecpagefont}{}
}

% Точки после верхних элементов оглавления
\renewcommand{\cftsecdotsep}{\cftdotsep}
\newcommand{\cftchapdotsep}{\cftdotsep}

\ifthenelse{\value{worktype} > 1}{%
    \renewcommand{\cftchapaftersnum}{.}
}{
\renewcommand{\cftsecaftersnum}{.}
\renewcommand{\cftsubsecaftersnum}{.}
}
%%% Списки (enumitem)

\usepackage {enumitem}      % Списки с настройкой отступов
\setlist %
{ %
  leftmargin = \parindent, itemsep=.5ex, topsep=.4ex
} %

% По ГОСТу нумерация должны быть буквами: а, б...
%\makeatletter
%    \AddEnumerateCounter{\asbuk}{\@asbuk}{м)}
%\makeatother
%\renewcommand{\labelenumi}{\asbuk{enumi})}
%\renewcommand{\labelenumii}{\arabic{enumii})}

%%% Таблицы: выбрать более подходящие

\usepackage{booktabs} % считаются наиболее профессионально выполненными
%\usepackage{ltablex}
%\newcolumntype {L} {>{---}l}

%%% Библиография

\usepackage{csquotes}        % Оформление списка литературы
\usepackage[
  backend=biber,
  hyperref=auto,
  language=auto,
  citestyle=gost-numeric,
  bibstyle=gost-numeric,
]{biblatex}
\addbibresource{biblio.bib} % Файл с лит.источниками

% Настройка величины отступа в списке
\ifthenelse{\value{worktype} < 2}{%
\defbibenvironment{bibliography}
  {\list
     {\printtext[labelnumberwidth]{%
    \printfield{prefixnumber}%
    \printfield{labelnumber}}}
     {\setlength{\labelwidth}{\labelnumberwidth}%
      \setlength{\leftmargin}{\labelwidth}%
      \setlength{\labelsep}{\dimexpr\MyIndent-\labelwidth\relax}% <----- default is \biblabelsep
      \addtolength{\leftmargin}{\labelsep}%
      \setlength{\itemsep}{\bibitemsep}%
      \setlength{\parsep}{\bibparsep}}%
      \renewcommand*{\makelabel}[1]{\hss##1}}
  {\endlist}
  {\item}
}{}

% ----------------------------------------------------------------
% Настройка переносов и разрывов страниц

\binoppenalty = 10000      % Запрет переносов строк в формулах
\relpenalty = 10000        %

\sloppy                    % Не выходить за границы бокса
%\tolerance = 400          % или более точно
\clubpenalty = 10000       % Запрет разрывов страниц после первой
\widowpenalty = 10000      % и перед предпоследней строкой абзаца

% ----------------------------
% Настройка подстветки ссылок
\hypersetup{
    colorlinks=true,       % false: boxed links; true: colored links
    linkcolor=red,          % color of internal links (change box color with linkbordercolor)
    citecolor=blue,        % color of links to bibliography
    filecolor=magenta,      % color of file links
    urlcolor=cyan           % color of external links
}

% Стили для окружений типа Определение, Теорема...
% Оформление теорем (ntheorem)

\usepackage [thmmarks, amsmath] {ntheorem}
\theorempreskipamount 0.6cm

\theoremstyle {plain} %
\theoremheaderfont {\normalfont \bfseries} %
\theorembodyfont {\slshape} %
\theoremsymbol {\ensuremath {_\Box}} %
\theoremseparator {:} %
\newtheorem {mystatement} {Утверждение} [section] %
\newtheorem {mylemma} {Лемма} [section] %
\newtheorem {mycorollary} {Следствие} [section] %

\theoremstyle {nonumberplain} %
\theoremseparator {.} %
\theoremsymbol {\ensuremath {_\diamondsuit}} %
\newtheorem {mydefinition} {Определение} %

\theoremstyle {plain} %
\theoremheaderfont {\normalfont \bfseries} 
\theorembodyfont {\normalfont} 
%\theoremsymbol {\ensuremath {_\Box}} %
\theoremseparator {.} %
\newtheorem {mytask} {Задача} [section]%
\renewcommand{\themytask}{\arabic{mytask}}

\theoremheaderfont {\scshape} %
\theorembodyfont {\upshape} %
\theoremstyle {nonumberplain} %
\theoremseparator {} %
\theoremsymbol {\rule {1ex} {1ex}} %
\newtheorem {myproof} {Доказательство} %

\theorembodyfont {\upshape} %
%\theoremindent 0.5cm
\theoremstyle {nonumberbreak} \theoremseparator {\\} %
\theoremsymbol {\ensuremath {\ast}} %
\newtheorem {myexample} {Пример} %
\newtheorem {myexamples} {Примеры} %

\theoremheaderfont {\itshape} %
\theorembodyfont {\upshape} %
\theoremstyle {nonumberplain} %
\theoremseparator {:} %
\theoremsymbol {\ensuremath {_\triangle}} %
\newtheorem {myremark} {Замечание} %
\theoremstyle {nonumberbreak} %
\newtheorem {myremarks} {Замечания} %


% Команды для использования в тексте работы


% макросы для начала введения и заключения
\newcommand{\Intro}{\addsec{Введение}}
\ifthenelse{\value{worktype} > 1}{%
    \renewcommand{\Intro}{\addchap{Введение}}%
}

\newcommand{\Conc}{\addsec{Заключение}}
\ifthenelse{\value{worktype} > 1}{%
    \renewcommand{\Conc}{\addchap{Заключение}}%
}

% Правильные значки для нестрогих неравенств и пустого множества
\renewcommand {\le} {\leqslant}
\renewcommand {\ge} {\geqslant}
\renewcommand {\emptyset} {\varnothing}

% N ажурное: натуральные числа
\newcommand {\N} {\ensuremath{\mathbb N}}

% значок С++ — используйте команду \cpp
\newcommand{\cpp}{C\nolinebreak\hspace{-.05em}%
\raisebox{.2ex}{+}\nolinebreak\hspace{-.10em}%
\raisebox{.2ex}{+}}

% Неразрывный дефис, который допускает перенос внутри слов,
% типа жёлто-синий: нужно писать жёлто"/синий.
\makeatletter
    \defineshorthand[russian]{"/}{\mbox{-}\bbl@allowhyphens}
\makeatother


\endinput

% Конец файла


\begin{document}

\begin{titlingpage}
    \begin{abstract}
      \centerline{\Large{\textbf{Аннотация}}}
      \vspace{16pt}
      Целью работы является развитие гибких и выразительных методов построения
      функциональных синтаксических анализаторов на основе современных подходов к контролю вычислительных эффектов. Производится сравнение двух подходов к управлению вычислительными эффектами, 
      трансформеров монад и расширяемых эффектов, с точки зрения эффективности 
      построения библиотек монадических комбинаторов парсеров. Разрабатываются 
      методы построения парсеров, основанных на этих подходах. 
      Производится построение транслятора подмножества~\lstinline{Markdown} 
      с возможностью \LaTeX"/вставок в~\lstinline{HTML}.

      \vspace{16pt}

      \textbf{Ключевые слова}: синтаксический анализ, парсер, монадические комбинаторы 
      парсеров, функциональное программирование, вычислительные эффекты.
    \end{abstract}
\end{titlingpage}

\tableofcontents

\Intro

\textbf{Актуальность темы.} Синтаксические анализаторы (парсеры) является необходимыми  
компонентами различных программных систем. Они используются для разбора исходного 
кода на языке программирования или разметки, с целью получения промежуточного 
представления для дальнейшей обработки данных. Парсеры могут быть сгенерированы 
автоматически с помощью специальных программ --- генераторов синтаксических 
анализаторов, или же написаны программистом в ручную. 

Синтаксические анализаторы, как и любые другие программные продукты, могут 
содержать ошибки реализации. Одним из методов повышения надёжности программных 
систем является использование для их разработки языков программирования со 
строгой статической типизацией и развитой системой типов. Современные функциональные 
языки программирования предоставляют возможность легковесной верификации программ
путём строгого контроля типов. 

Функциональное программирование --- парадигма программирования, которая трактует
программу как вычисление значения некоторой математической функции. 
Функциональный стиль построения парсеров заключается в том, что парсер
представляется функцией, которая принимает на вход поток символов и строит
по нему некоторое абстрактное синтаксическое дерево. Удобно считать, что не каждый
парсер полностью потребляет входной поток, это даёт возможность
построения результирующего парсера по частям, из примитивных. Также необходимо
иметь средства для обработки некорректного входа. Для этого существуют разные
способы: можно сообщать об ошибке разбора, либо заменять неудачу списком
успехов~\cite{wadlerSuccess}. Высказанные пожелания о характеристиках
парсера могут быть отражены следующим типом языка Haskell:

\begin{figure}[h]
\begin{lstlisting}
type Parser a = String -> [(a,String)]
\end{lstlisting}
\caption{Тип Parser}
\end{figure}

Типы вида~\lstinline{Parser a} можно рассматривать как вычисления с определённым 
побочным эффектом. Для расширения возможностей и повышения удобства синтаксического 
анализа можно расширять набор эффектов функций-парсеров.    
Современные исследования в области функционального программирования предоставляют
различные методы типизации вычислений с побочными эффектами. 
Актуальной задачей является апробация этих абстракций на практических
задачах. Синтаксический анализ традиционно считается одной из модельных
задач в мире функционального программирования и хорошо подходит для практического
применения новых абстракций.

\textbf{Объектом исследования} является теория построения синтаксических анализаторов 
с использованием статически типизированных функциональных языков программирования.  

\textbf{Предметом исследования} является методы функционального программирования, 
предназначенные для описания вычислений с несколькими побочными эффектами, 
в приложении к построению синтаксических анализаторов.  

\textbf{Целью работы} является развитие гибких и выразительных методов построения 
синтаксических анализаторов, на основе современных подходов к контролю вычислительных
эффектов.
К достижению этой цели приводит решение следующих \textbf{задач}:
\begin{enumerate}
  \item Разработка библиотеки монадических комбинаторов парсеров, основанных на 
  трансформерах монад.
  \item Разработка библиотеки монадических комбинаторов парсеров, основанных на 
  расширяемых эффектах.
  \item Разработка транслятора подмножества языка~\lstinline{Markdown} с возможностью
  \LaTeX"/вставок в~\lstinline{HTML}.
\end{enumerate}  

В качестве отправной точки для разработки библиотеки монадических комбинаторов
парсеров используются результаты статьи~\cite{monParsing}. Описание вычислений
с несколькими побочными эффектами производится с помощью трансформеров монад
~\cite{monadTransformers}, а также расширяемых эффектов~\cite{extEffects} ---
альтернативного трансформерам монад подхода. Для повышения гибкости разрабатываемых 
методов и построения полиморфных по входному потоку парсеров применяются 
специализированные моноиды, представленные в статье~\cite{monoids}.


\chapter{Обзор подходов к управлению вычислительными эффектами и построению
монадических парсеров}

\section{Вычисления с побочными эффектами}

  Возможность статического контроля побочных эффектов является одним из принципиальных 
  преимуществ статически типизированных языков программирования с богатыми системами типов. 
  Отделение ``чистых'' функций от вычислений, обременённых взаимодействием с 
  устройствами ввода-вывода и другими побочными эффектами, позволяет с большей
  уверенностью рассуждать о надёжности разрабатываемой программы. 

  Далее будет произведён обзор и сравнение двух подходов к управлению вычислительными 
  эффектами.   

\section{Сравнение трансформеров монад и расширяемых эффектов}
 
  В данной работе главными параметрами, относительно которых производилось сравнение, 
  выступали абстрактность и гибкость, то есть: насколько общим является тот или иной подход к управлению вычислительными эффектами, а также насколько удобно его можно подогнать к 
  специфической задаче, в данном случае, задаче синтаксического анализа.  

  \subsection{Трансформеры монад}

    В статье~\cite{monadTransformers} описаны объекты, трансформеры монад,
    которые можно использовать в качестве блоков для построения типов, описывающих
    вычисления с побочными эффектами. Каждый из трансформеров монад позволяет
    добавить некоторый вычислительный эффект к внутренней монаде, при
    этом для результирующего типа также возможно построение экземпляра класса
    типов \lstinline{Monad}.

    На данный момент трансформеры монад являются распространённым способом
    построения вычислений с несколькими побочными эффектами. Однако эта
    техника имеет недостатки: проблема невозможности автоматического
    \emph{подъёма} (англ. \lstinline{lift}) при наличии в стеке двух эффектов одного
    рода, связанная с первой проблема статически определённого порядка эффектов в
    стеке, а также невозможность скомбинировать две произвольные монады.

    В листинге~\ref{listing:mtlReadersLift} приведён пример монадической функции,
    которая должна иметь два различных конфигурационных параметра. Если не
    произвести явный подъём, то компиляция этой функции завершится с ошибкой
    проверки типов вида~\ref{listing:mtlCompileError}: компилятор не в состоянии
    самостоятельно вывести нужный тип для функции \lstinline{ask}.

    \begin{figure}[t]
    \begin{lstlisting}
    adder :: ReaderT String (Reader Int) Int
    adder = do
      str <- ask
      num <- lift ask
      return $ num + read str

    runnerForAdder = runReader (runReaderT adder "2") 3
    \end{lstlisting}
    \caption{Необходимость явного подъёма во внутреннюю монаду}
    \label{listing:mtlReadersLift}
    \end{figure}

    Порядок монад в стеке определён статически и закодирован в типе функции,
    в листинге~\ref{listing:mtlDifferentReadersLift} представлена та же функция,
    эквивалентная по смыслу, но имеющая другой тип. Значительной деталью также
    является тот факт, что в реализации этой функции \lstinline{lift} применяется
    к другому вызову функции \lstinline{ask}.

    \begin{figure}[t]
    \begin{lstlisting}
    adder :: ReaderT Int (Reader String) Int
    adder = do
      str <- lift ask
      num <- ask
      return $ num + read str

    runnerForAdder = runReader (runReaderT adder 3) "2"
    \end{lstlisting}
    \caption{Функция из листинга~\ref{listing:mtlReadersLift} с другим порядком монад в стеке}
    \label{listing:mtlDifferentReadersLift}
    \end{figure}

    Функция \lstinline{runnerForAdder} из листингов~\ref{listing:mtlReadersLift}
    и~\ref{listing:mtlDifferentReadersLift}, запускающая вычисление, полностью
    определяется типом вычисления, она производит развёртку стека монад.
    В следующем подразделе будет рассмотрено средство комбинирования вычислительных
    эффектов, при использовании которого порядок на наборе эффектов устанавливается
    именно функцией, запускающей вычисление, а сам набор является неупорядоченным.

    \begin{figure}[t]
    \begin{lstlisting}
    No instance for (MonadReader Int (ReaderT String (Reader Int)))
          arising from a use of ask
    \end{lstlisting}
    \caption{Ошибка типов при отсутствии явного подъёма}
    \label{listing:mtlCompileError}
    \end{figure}

    Следует сделать замечание относительно предыдущего примера: разнородная
    конфигурационная информация могла быть объединена в алгебраический тип данных,
    и тогда нужда в использовании двух эффектов \lstinline{Reader} отпала бы,
    однако такое решение может не сработать для других эффектов, поэтому необходимо
    искать универсальный способ.

    Ещё одной проблемой трансформеров монад является необходимость описывать 
    \lstinline{n} экземпляров классов типов, 
    по количеству уже существующих трансформеров, при введении нового трансформера.  
    Предположим, что возникла необходимость ввести класс типов для некоторого нового 
    вычислительного эффекта: листинг~\ref{listing:mtlNewEffect}. 

    \begin{figure}[t]
    \begin{lstlisting}
    class Monad m => MonadNew a m where
      action1 :: m a
      action2 :: m ()
    \end{lstlisting}
    \caption{Класс типов для описания возможностей нового вычислительного эффекта}
    \label{listing:mtlNewEffect}
    \end{figure}

    Для того, чтобы полноценно использовать новый эффект, необходимо задать 
    правила ``поднятия'' (\lstinline{lift}), то есть, описать экземпляры классов типов для 
    всех трансформеров из~\lstinline{mtl}. В листинге~\ref{listing:mtlNewEffectInstances}
    приведены примеры экземпляров классов типов.

    \begin{figure}[t]
    \begin{lstlisting}
    instance MonadNew a m => MonadNew a (IdentityT m) where
      action1 = lift action1
      action2 = lift action2

    instance MonadNew a m => MonadNew a (MaybeT m) where
      action1 = lift action1
      action2 = lift action2

    ...

    \end{lstlisting}
    \caption{Экземпляры классов типов для всех возможных комбинаций нового эффекта 
    с существующими}
    \label{listing:mtlNewEffectInstances}
    \end{figure}

    Ведутся активные поиски способов комбинирования монад, которые были бы лишены
    вышеперечисленных недостатков. Одним из развивающихся направлений являются
    расширяемые эффекты (\lstinline{Extensible Effects}), которе будут обсуждаться
    в следующем разделе.

  \subsection{Расширяемые эффекты}

    Расширяемые эффекты, подробно описанные в статье~\cite{extEffects}, представляют
    собой альтернативный трансформерам монад подход к описанию вычислений с поблочными
    эффектами.

    Суть подхода заключается в аналогии между вычислительными эффектами и
    клиент-серверным взаимодействием. Код, который собирается породить некоторый
    побочный эффект: совершить ввод-вывод, бросить исключение и тому подобное должен
    отправить \emph{запрос} на обработку этого эффекта особой глобальной сущности
    --- менеджеру эффектов. Запрос описывает действие, которое необходимо
    произвести, а также функцию-продолжение (англ. \emph{continuation}) для
    возобновления работы после обработки запроса.

    В ранних разработках, относящихся к такому подходу, менеджер запросов не был
    частью программы пользователя, а являлся отдельной сущностью, подобно ядру
    операционной системы, или обработчику \lstinline{IO}-действий в
    \lstinline{Haskell}. Этот глобальный внешний авторитет имел контроль над всеми
    ресурсами (файлами, памятью и другими): он обрабатывал запрос и принимал решение:
    исполнить его и продолжить выполнение запросившего кода, либо остановить
    вычисление и вернуть результат. При таком подходе нет никакой необходимости в
    указании явного порядка при комбинировании эффектов, однако недостатком является
    негибкость внешнего интерпретатора эффектов, кроме того, эффекты никак
    не отражаются в типах.

    Разработчики библиотеки~\lstinline{Extensible Effects} модифицировали концепцию:

    \begin{itemize}
      \item
    Глобальный обработчик запросов был заменён на средство, которое является частью
    пользовательской программы, некий аналог обработчиков исключений: теперь вместо
    единого менеджера для всех эффектов существуют локальные обработчики для каждого
    типа эффектов, такой подход называется \emph{алгебраическими
    обработчиками}~\cite{effAndHandl}. Каждый такой обработчик является менеджером
    для соответствующей клиентской части программы, а также и клиентом сам по себе:
    он пересылает запросы, которые не может обработать, менеджеру верхнего уровня.
      \item
    Разработана выразительная система типов-эффектов, которая отслеживает какие
    эффекты активны в данном вычислении. Эта система поддерживает особую структуру
    данных: \emph{открытое объединение} (\lstinline{Open Union}, индексированное
    типами копроизведение функторов), содержащее неупорядоченный набор
    вычислительных эффектов. Действие каждого обработчика отражается в типе:
    происходит удаление из набора эффекта, который был обработан. Таким
    образом система типов может отследить, все ли эффекты обработаны.
      \item
    Синтаксис для работы с эффектами построен по аналогии с синтаксисом для
    трансфомеров монад. Код, написанный с использованием трансформеров монад,
    может быть переведён на расширяемые эффекты с минимальными синтаксическими
    изменениями.
    \end{itemize}

    Есть два способа обозначить принадлежность эффекта \lstinline{m} открытому
    объединению \lstinline{r}. С помощью типового ограничения
    \mbox{\lstinline{Member m r},} которое означает, что вычисления имеет
    \emph{по крайней мере} один побочный эффект \lstinline{m}. Иначе, можно явно
    указать шаблон \lstinline{m :> r'} разложения набора \lstinline{r} на
    эффект \lstinline{m} и оставшийся набор \lstinline{r'}, что аналогично
    теоретико-множественному обозначению $\{$\lstinline{m}$\}$ $\cup$
    \lstinline{r'}. Тип \lstinline{Void} играет роль $\varnothing$, то есть
    вычисление с типом \lstinline{Eff Void a} является чистым, а вычисления с
    типом \lstinline{Eff (Reader Int :> Reader Bool :> Void)} может иметь побочный
    эффект обращения к двум средам конфигурационной информации.

    Авторы также указывают, что ограничение
    \mbox{\lstinline{Member (Reader Int) r}} выглядит похоже на ограничение
    принадлежности классу типов \lstinline{MonadReader}, более того, возможно
    объявить экземпляр этого класса для монады \lstinline{Eff r}. Однако в этом нет
    необходимости, так как тип \lstinline{Eff r} является более выразительным:
    в листинге~\ref{listing:extEff2Readers} приводится пример работы с функцией,
    имеющей два эффекта \lstinline{Reader}. Эта функция демонстрирует одно из
    преимуществ расширяемых эффектов перед трансформерами монад: пропадает
    необходимость явного вызова функции \lstinline{lift}, которую можно наблюдать
    в листингах~\ref{listing:mtlReadersLift}
    и~\ref{listing:mtlDifferentReadersLift}. Каждое вхождение функции
    \lstinline{ask} обращается к своей собственной среде, определяемой типом.

    \begin{figure}[t]
    \begin{lstlisting}
    adder :: ( Member (Reader Int) r
             , Member (Reader String) r
             ) => Eff r Int
    adder = do
      num <- ask
      str <- ask
      return $ num + read str

    runAdder = run $ runReader (runReader adder "2") (1 :: Int)
    \end{lstlisting}
    \caption{Пример функции с двумя средами конфигурации.}
    \label{listing:extEff2Readers}
    \end{figure}

    Другое важное отличие расширяемых эффектов от трансформеров монад ---
    неупорядоченность набора эффектов до запуска вычисления --- никак не проявило
    себя в примере из листинга~\ref{listing:extEff2Readers}, потому что были
    использованы одинаковые эффекты. В листинге~\ref{listing:extEffOrdering}
    рассмотрен синтетический (для краткости) пример, демонстрирующий динамическую
    установку порядка на множестве эффектов. Функция из примера считает до нуля и
    завершается с исключением. Если запускать вычисление в порядке, представленном
    в \lstinline{runCountdown1}, то результатом будет \lstinline{Nothing},
    сообщающий об исключении. При запуске в порядке \lstinline{runCountdown2},
    результатом будет пара \lstinline{(0,Nothing)} из последнего состояния и
    сообщения об исключении.

    \begin{figure}[t]
    \begin{lstlisting}
    countdown :: ( Member Fail r
                 , Member (State Int) r
                 ) => Eff r ()
    countdown = do
      state <- get
      if state == (0 :: Int)
      then die
      else put (state - 1) >> countdown

    runCountdown1 n = run $ runFail $ runState (n :: Int) $ countdown

    runCountdown2 n = run $ runState (n :: Int) $ runFail $ countdown
    \end{lstlisting}
    \caption{Порядок на множестве эффектов определяется динамически}
    \label{listing:extEffOrdering}
    \end{figure}

  \subsection{Резюме}

    Оба описанных выше подхода имеют свои достоинства и недостатки. Так, концептуально, 
    расширяемые эффекты являются более прогрессивным и гибким методом контроля 
    вычислений с побочными эффектами, главным образом за счёт возможности использовать   
    несколько однородных эффектов без потери автоматического поднятия на нужный уровень, 
    а также отсутствии необходимости в явном описании огромного количества тривиальных
    экземпляров классов типов для обеспечения механизма ``поднятия''.  
    Однако, расширяемые эффекты ещё не успели получить такого широкого распространения, 
    как трансформеры монад, в следствии чего их нельзя считать готовыми к использованию 
    в критических приложениях. 

    Трансформеры монад же, в свою очередь, являются зрелой концепцией, имеющей несколько
    реализаций в виде Haskell-модулей, и использующейся в огромном количестве кода. 

\section{Монадические функциональные парсеры}

  Популярной библиотекой монадических парсеров является библиотека
  \lstinline{Parsec}~\cite{parsec}. Это библиотека промышленного уровня,
  используемая во многих проектах, например в универсальном конвертере документов
  \lstinline{Pandoc}~\cite{pandoc}. Преимуществами \lstinline{Parsec} являются
   его гибкость и подробность сообщений об ошибках.

  Другой популярной библиотекой является
  \lstinline{attoparsec}~\cite{attoparsec}. При разработке
  \lstinline{attoparsec} акцент был сделан на скорость, поэтому было принято
  решение ограничиться только одним типом \lstinline{ByteString} и пожертвовать
  информативностью сообщений об ошибках. Основным предназначением
  \lstinline{attoparsec} является анализ сетевых протоколов.

  Недостатками обеих библиотек является недостаточная абстрактность в плане типов
  данных для представления входного потока. Применение классов типов,
  представленных в предыдущем подразделе, позволит получить абстрактный код
  способный работать с любыми строковыми типами.

  Архитектура обеих библиотек использует концепцию трансформеров монад для
  представления вычислений с несколькими побочными эффектами. Необходимо
  исследовать другие концепции многоэффектных вычислений в приложении к
  монадическим парсерам.

\section{Выбранные для реализации практических задач методы}

Для реализации поставленных практических задач использовалась концепция 
трансформеров монад~\cite{monadTransformers} и основанная на ней библиотека
\lstinline{MTL}~\cite{mtlHackage}, расширяемые эффекты~\cite{extEffects} 
и библиотека \lstinline{extensible-effects}~\cite{extensibleEffectsHackage} а также библиотека
\lstinline{monoid-subclasses}~\cite{monoidSubclassesHackage}.


\chapter{Теоретические основы функционального программирования и элементы теории 
         категорий}

Функциональное программирование --- парадигма программирования, которая трактует 
программу как вычисление значения некоторой математической функции.

Корни функционального программирования уходят в $\lambda$"/исчисление,
формальную систему, разработанную в 1930-х годах для решения
Entscheidungsproblem~\cite{entscheidungsproblem}.

Одним из главных преимуществ функциональных языков программирования считается
высокий уровень абстракции, выразительность и высокий коэффициент повторного
использования кода. Для достижения этих свойств в языках реализуются такие
средства как параметрический полиморфизм, функции высших порядков, каррирование,
алгебраические типы данных, классы типов и другие. Использование неизменяемых
данных позволяет существенно упростить отладку программ.

\section{Классы типов}

  Классы типов позволяют накладывать ограничения на типы, определяя некоторый
  набор операций, которые могут производиться над типами, принадлежащими к
  некоторому классу. Иными словами, класс типов определяет интерфейс,
  через который можно взаимодействовать с типом.

  Рассмотрим определение стандартного класса типов языка \lstinline{Haskell},
  отвечающего за возможность сравнения на равенство:

  \begin{figure}[h]
  \begin{lstlisting}
  class Eq a where
    (==) :: a -> a -> Bool
    (/=) :: a -> a -> Bool
  \end{lstlisting}
  \caption{Класс типов, для которых введено отношение эквиваленции}
  \label{listing:Eq}
  \end{figure}

  Чтобы воспользоваться возможностями класса типов для какого-то конкретного типа
  необходимо объявить этот тип экземпляром класса типов, во многих случаях это
  может быть сделано неявно, благодаря автоматическому порождению экземпляров
  компилятором \lstinline{GHC}.

  Возможность автоматического порождения экземпляра нужного класса типов для
  алгебраического типа данных:
  \begin{figure}[h]
  \begin{lstlisting}
  data Numbers = One | Two | Three deriving Eq

  ghci> One /= Two
  True
  \end{lstlisting}
  \caption{Автоматическое порождение экземпляров классов типов}
  \label{listing:List}
  \end{figure}


  В некоторых случаях необходимо описать экземпляр явно, такая возможность тоже
  существует.

  \begin{figure}[h]
  \begin{lstlisting}
  data Foo = Foo {x :: Integer, str :: String}

  instance Eq Foo where
    (Foo x1 str1) == (Foo x2 str2) = (x1 == x2) &&
                                     (str1 == str2)
  \end{lstlisting}
  \caption{Явное описание экземпляра класса типов}
  \label{listing:Instance}
  \end{figure}

  Как уже говорилось ранее, код, написанный на языке программирования
  \lstinline{Haskell} имеет очень высокий уровень абстракции. Далее будут
  рассмотрены два классы типов, представляющие структуры, введённые в язык
  \lstinline{Haskell} под влиянием исследований в области связи теоретической
  информатики и теории категорий. Эти структуры представляют абстракцию для
  вычислений с побочными эффектами. В дальнейшем изложении эти классы типов будут
  использоваться для построения парсеров.

\section{Функтор}

  Рассмотрим определение класса типов \lstinline{Functor} из стандартной
  библиотеки \lstinline{Haskell}:

  \begin{figure}[h]
  \begin{lstlisting}
    class Functor f where
      fmap :: (a -> b) -> f a -> f b
  \end{lstlisting}
  \caption{Класс типов \lstinline{Functor}}
  \label{listing:Functor}
  \end{figure}

  Как видно из определения, каждый тип, являющийся функтором, должен предоставлять
  функцию \lstinline{fmap}.

  Сформировались два основных неформальных описания функтора: контейнер,
  содержащий в себе значения определённого типа и вычисление, которое производится
  в некотором контексте. Согласно первой из этих трактовок, функция
  \lstinline{fmap} применяет подаваемую ей на вход функцию к значениям в
  контейнере и возвращает изменённые значения, с сохранением структуры контейнера.
  Если же говорить в терминах вычислительных контекстов, то \lstinline{fmap}
  модифицирует вычисление в контексте, но сам контекст остаётся неизменным.

  Упомянутые выше списки языка \lstinline{Haskell} являются функторами, для них в
  качестве \lstinline{fmap} можно выбрать функцию \lstinline{map}.

  Важно заметить, что не любой тип, для которого определена функция
  \lstinline{fmap} является функтором, необходимо также потребовать выполнения
  двух уравнений, называемых законами функтора:

  \begin{figure}[h]
  \begin{lstlisting}
  fmap id = id
  fmap (g . h) = (fmap g) . (fmap h)
  \end{lstlisting}
  \caption{Законы функтора}
  \label{listing:FunctorLaws}
  \end{figure}

  Здесь \lstinline{id} --- тождественная функция, а \lstinline{.} --- инфиксный
  оператор композиции функций. Эти уравнения отражают теоретико-категорное
  определение функтора как отображения между категориями, сохраняющее единичный
  морфизм и композицию. В случае \lstinline{Haskell} рассматриваются эндофункторы
  над категорией \lstinline{Hask} --- категорией типов языка \lstinline{Haskell}.

\section{Аппликативный функтор}

  Аппликативный функтор является специализацией функтора, допускающей применение
  функции, находящейся внутри некоторого контекста, к значению в таком же
  контексте. Аппликативные функторы также иногда называют \emph{идиомами}
  (англ. idioms).

  В языке \lstinline{Haskell} для представления аппликативных функторов
  используется стандартный класс типов \lstinline{Applicative},
  описанный в модуле \lstinline{Control.Applicative}.

  \begin{figure}[h]
  \begin{lstlisting}
  class (Functor f) => Applicative f where
      pure :: a -> f a
      (<*>) :: f (a -> b) -> f a -> f b
  \end{lstlisting}
  \caption{Класс типов \lstinline{Applicative}}
  \label{listing:Applicative}
  \end{figure}

  Для всякого типа, который является аппликативным функтором, должны выполняться
  законы, представленные в листинге~\ref{listing:ApplicativeLaws}.

  \begin{figure}[h]
  \begin{lstlisting}
  pure id <*> v = v

  pure (.) <*> u <*> v <*> w = u <*> (v <*> w)

  pure f <*> pure x = pure (f x)

  u <*> pure y = pure ($ y) <*> u
  \end{lstlisting}
  \caption{Законы аппликативного функтора}
  \label{listing:ApplicativeLaws}
  \end{figure}

  Любая монада является аппликативным функтором, но не каждый аппликативный
  функтор является монадой. То есть, интерфейсу класса типов
  \lstinline{Applicative} удовлетворяет больше типов, чем интерфейсу класса типов
  \lstinline{Monad}.

  С понятием аппликативного функтора связан особый синтаксис описания вычислений с
  эффектами на \lstinline{Haskell}, называемый \emph{аппликативным стилем}. Этот
  стиль имеет преимущество относительно \lstinline{do}-нотации для монад ---
  компилятор имеет возможность генерировать для него более оптимальный код.
  Многие монадические функции могут быть переписаны в аппликативном стиле.
  В листинге~\ref{listing:MonadApplicative} представлена короткая монадическая
  функция и эквивалентная ей аппликативная, задача которых прочитать из потока
  стандартного ввода две строки выполнить их конкатенацию и вернуть результат.

  \begin{figure}
  \begin{lstlisting}
  action :: IO String
  action = do
    a <- getLine
    b <- getLine
    return $ a ++ b

  action :: IO String
  action = (++) <$> getLine <*> getLine
  \end{lstlisting}
  \caption{\lstinline{do}-нотация и аппликативный стиль}
  \label{listing:MonadApplicative}
  \end{figure}

\section{Монада}

  Вершиной иерархии типов, описывающих вычисления с побочными эффектами,
  являются монады. Возникнув как средство для введения в чистый функциональный
  язык возможности выполнять операции ввода-вывода, монады были обобщены и на
  другие вычислительные эффекты, для которых есть необходимость в построении
  композиции функций, и теперь являются наиболее известной абстракцией такого
  сорта.

  Для представления монад в языке \lstinline{Haskell} используется класс типов
  \lstinline{Monad}. Кроме того, существует три закона монады, которые должны
  выполняться для каждого типа, имеющего экземпляр класса типов \lstinline{Monad}.
  Эти законы отражают тот факт, что монада является моноидом в категории
  эндофункторов. Важно знать, что система типов не гарантирует выполнения
  монадических законов, ответственность за них целиком лежит на программисте,
  который разрабатывает экземпляр \lstinline{Monad} для некоторого типа.

  \begin{figure}[h]
  \begin{lstlisting}
  class Monad m where
    (>>=) :: m a -> (a -> m b) -> m b
    (>>) :: m a -> m b -> m b
    return :: a -> m a
  \end{lstlisting}
  \caption{Класс типов \lstinline{Monad}}
  \label{listing:Monad}
  \end{figure}

  \begin{figure}[h]
  \begin{lstlisting}
  return a >>= k = k a
  m >>= return = m
  m >>= (\x -> k x >>= h) = (m >>= k) >>= h
  \end{lstlisting}
  \caption{Законы монады}
  \label{listing:MonadLaws}
  \end{figure}

  Монада естественным образом возникает при построении функциональных парсеров,
  которые описываются в третьей главе.

\section{Копроизведение}

  Копроизведением объектов $A$ и $B$ в категории называется объект $A + B$ и две
  стрелки $i_1 : A \to A + B$ и $i_2 : B \to A + B$ такие, что для всех стрелок
  $f : A \to C$ и $g : B \to C$ существует единственная стрелка
  $\langle f|g\rangle  : A + B \to C$ для которой следующая диаграмма
  коммутативна~\cite{TeorCat}.

  \begin{center}
  \begin{tikzcd}
  A \arrow{r}{f} \arrow{rd}[swap]{i_1}
  &C
  &B \arrow{ld}{i_2} \arrow{l}[swap]{g}\\
  &A+B \arrow{u}[description]{\langle f|g\rangle}
  \end{tikzcd}
  \end{center}

  Стрелки $i_1$ и $i_2$ называются каноническими вложениями.

  В случае категории множеств копроизведением является дизъюнктивное
  или размеченное объединение.

  Понятие копроизведения, в случае категории типов языка \lstinline{Haskell},
  потребуется в главе 2, для описания расширяемых эффектов.

\section{Моноиды в функциональном программировании}

  Моноидом является множество $M$ с заданной на нём бинарной ассоциативной
  операцией $*$, и в котором существует такой элемент
  $e$, что $\forall x \in M~e*x = x*e = x$. Элемент $e$ называется единицей.

  В языке \lstinline{Haskell} для представления моноидов существует специальный
  класс типов.

  \begin{figure}[h]
  \begin{lstlisting}
  class Monoid a where
    mempty  :: a
    mappend :: a -> a -> a
  \end{lstlisting}
  \caption{Моноид в \lstinline{Haskell}}
  \label{listing:Monoid}
  \end{figure}

  Cтроки с операцией конкатенации и пустой строкой в качестве единицы являются
  моноидом. С помощью класса типов \lstinline{Monoid} возможна абстрактная работа
  с любыми строковыми типами. Однако существуют ситуации, в которых интерфейса
  класса типов \lstinline{Monoid} оказывается недостаточно, возникает
  необходимость в функции, отделяющей префикс, аналогичной функции
  \lstinline{head} для списков. Для таких целей создана библиотека
  \lstinline{monoid-subclasses}~\cite{monoids}, предоставляющая
  классы типов, являющиеся моноидами специального вида, которые допускают
  необходимые дополнительные операции. В случае парсеров такой дополнительной
  операцией является аналог операции разделения списка на голову и хвост.

  \begin{figure}[h]
  \begin{lstlisting}
  splitCharacterPrefix :: t -> Maybe (Char, t)

  uncons :: [a] -> Maybe (a, [a])
  \end{lstlisting}
  \caption{Отделение атомарного префикса и взятие головы списка}
  \label{listing:Monoid}
  \end{figure}

\chapter{Methods of parser construction}

Consider a simple type to represent a parser (listing~\ref{listing:basicParserType}). 
In this representation, parser is a
function, taking input stream and returning a list of possible valid
variants of analysis in conjunction with corresponding input stream 
remains. Empty list of result stands for completely unsuccessful attempt of 
parsing, whereas multiple results mean ambiguity.

\begin{figure}[h]
\begin{lstlisting}
type Parser a = String -> [(a,String)]
\end{lstlisting}
\caption{Type of Parser}
\label{listing:basicParserType}
\end{figure}

Types similar to $Parser a$ may be treated as effectful computation. In this 
particular example, effect of non-determinism is exploited to express ambiguity 
of parsing. To represent computations with effects a concept of Monad is used in 
~\lstinline{Haskell} programming language. Comprehensive information about 
properties of parsers like~\ref{listing:basicParserType} may be found in 
paper~\cite{monParsing}.

To extend capabilities and improve convenience of syntactic analysers set of 
effects of parsers could be expanded: it is handy to run parsers in a configurable 
environment or introduce logging. In this section two approaches to combination 
of computational effects will be considered: monad transformers and extensible 
effects.

\section{Parser as a monad transformer stack}

  Monad transformer is a concept which lets to enrich a given monad with a 
  property of other monad. Multiple monad transformers may be combined 
  together to form monad stack, that is, a monad possessing all properties of 
  it's components.

  Papers~\cite{monParsing} proposes a way of decomposition of type from 
  listing~\ref{listing:basicParserType} into stack of two monads: state and list,
  where the last one provides effect of non-determinism. Thus, type for parser 
  takes a form introduced in listing~\ref{listing:monadStackParserType}.

  \begin{figure}[h]
  \begin{lstlisting}
  type Parser a = StateT String [] a
  \end{lstlisting}
  \caption{Parser as a simple monad stack}
  \label{listing:monadStackParserType}
  \end{figure}

  Parser combinator library developed in this work uses following monad 
  stack~\ref{listing:hugeMonadStackParserType}. This representation of a 
  parsers also is parametrised with type of input stream. 
  Types~\lstinline{ParserState} and~\lstinline{ErrorReport} are algebraic
  data types for representing parser's state and possible analysis errors 
  respectively.

  \begin{figure}[h]
  \begin{lstlisting}
  newtype Parser t a = Parser (
      StateT (ParserState t) (Either (ErrorReport t)) a
    ) deriving ( Functor, Applicative, Monad
               , MonadState (ParserState t)
               , MonadError (ErrorReport t)
               )
  \end{lstlisting}
  \caption{Parser as a stack of two monads}
  \label{listing:hugeMonadStackParserType}
  \end{figure}

  The most low-level primitive which serves as a basis for all parser combinators 
  is a parser that consumes a single item from input stream 
  (see listing~\ref{listing:mtlParsersItem}).

  \begin{figure}[h]
  \begin{lstlisting}
  item :: TM.TextualMonoid t => Parser t Char
  item = do
    state  <- get
    let s = TM.splitCharacterPrefix . remainder $ state
    case s of
      Nothing -> throwError (EmptyRemainder "item",state)
      Just (c,rest) -> do
        let (c,rest) = fromJust s
        put (ParserState {position = updatePos (position state) c, remainder = rest})
        return c
  \end{lstlisting}
  \caption{Single input item consuming parser}
  \label{listing:mtlParsersItem}
  \end{figure}

  Listings~\ref{listing:mtlParserSat} and~\ref{listing:mtlParserString}
  contain two more parsers from developed library: conditional consumer and 
  given string consumer.

  \begin{figure}[h]
  \begin{lstlisting}
  sat :: TM.TextualMonoid t => (Char -> Bool) -> Parser t Char
  sat p = do
    state <- get
    x <- item `overrideError` (EmptyRemainder "sat")
    if p x then return x else
      throwError (UnsatisfiedPredicate "general",state)
  \end{lstlisting}
  \caption{}
  \label{listing:mtlParserSat}
  \end{figure}

  \begin{figure}[h]
  \begin{lstlisting}
  string :: TM.TextualMonoid t => String -> Parser t String
  string s = do
    state <- get
    (mapM char s) `overrideError`
      (UnsatisfiedPredicate ("string " ++ s))
  \end{lstlisting}
  \caption{}
  \label{listing:mtlParserString}
  \end{figure}  

  To actually perform parsing, it's necessary to implement a function that 
  ``runs'' a computation (listing~\ref{listing:mtlParseRun}). It's need to be
  pointed out, that order of effect handling is statically encoded in type of 
  monad stack. 

  \begin{figure}[h]
  \begin{lstlisting}
parse :: TM.TextualMonoid t =>
  Parser t a -> t -> Either (ErrorReport t) (a,ParserState t)
parse (Parser p) s =
  runStateT p (ParserState {remainder = s, position = initPos})
    where initPos = (1,1)
  \end{lstlisting}
  \caption{}
  \label{listing:mtlParseRun}
  \end{figure}

  Overall, a concept of has a considerable convenience in programming due to 
  its maturity and popularity. However, as it was discussed in section 1, this 
  approach lacks flexibility, doesn't allow stacks with several homogeneous 
  effects (for instance, multiple~\lstinline{StateT} transformers) without 
  loosing automatic lifting (~\lstinline{lift}) and requires boilerplate 
  typeclass instance declaration.

  Next, different method of monadic parser combinators will be considered: one 
  based on extensible effects --- an alternative framework of construction of 
  effectful computation.

\section{Parsers based on extensible effects}

  Extensible effects, presented in paper~\cite{extEffects}, are an alternative 
  to monad transformers approach to describing effectful computation.

  An idea behind extensible effects, in a nutshell, is all about analogy between
  client-server interaction and computational effects. Commands of code is about 
  to produce some side-effect such as IO, exception, etc. have to send a 
  \emph{request} for handling this effect to a special authority --- an effect 
  manager. Request describes an action that should be performed alongside with 
  a continuation.

  Listing~\ref{listing:extEffItem} contains basic primitive of the library --- 
  function that consumes a single item of input stream. Type annotation of this 
  function declares effects performed by this function: fallible computation 
  and presence of state. Let us take a closer view on this type annotation.
  Constraint~\lstinline{Member Fail r} points out that set of effects~\lstinline{r}
  must contain effect~\lstinline{Fail}, whereas type of return 
  value~\lstinline{Eff r Char} tells that function~\lstinline{item} yields value
  of type~\lstinline{Char} and may perform effects from set~\lstinline{r}. 

  \begin{figure}[h]
  \begin{lstlisting}
item :: ( Member Fail r
        , Member (State String) r
        ) => Eff r Char
item = do
  s <- get
  case s of
    [] -> die
    (x:xs) -> put xs >> return x
  \end{lstlisting}
  \caption{Single input item consuming parser}
  \label{listing:extEffItem}
  \end{figure}

  Generally, from syntactic point of view, declaration of combinators based on 
  extensible effects is similar to regular monadic code. This is achieved by 
  type~\lstinline{Eff r a} having an instance of~\lstinline{Monad} typeclass.
  \lstinline{Eff r a} is a free monad constructed on top of functor~\lstinline{r}
  which is a open union of effects. As long as~\lstinline{Eff r a} is a monad, 
  regular monadic do-notation and applicative style become available.  

  \begin{figure}[h]
  \begin{lstlisting}
sat :: ( Member Fail r
       , Member (State String) r
       ) => (Char -> Bool) -> Eff r Char
sat p = do
  (s :: String) <- get
  x <- item
  if p x then return x else (put s >> die)
  \end{lstlisting}
  \label{listing:extEffSat}
  \end{figure}

  Extensible effects, in contrast to monad transformers, allow to set an order of 
  effect handling just before running computation. Thus, same computation may 
  produce different behaviour, controlled by order of effect application of 
  handlers. For instance, in listing~\ref{listing:extEffparse} types of handlers 
  ~\lstinline{parse} and~\lstinline{parse'} are different because~\lstinline{parse} 
  handles~\lstinline{Fail} after~\lstinline{State} and yields pair of last occurred 
  state and possibly missing result of parsing, i.e. saves last state with no respect 
  to success of parsing. Conversely,~\lstinline{parse'} handles~\lstinline{State}
  first and doesn't return any state in case of unsuccessful parsing.

  \begin{figure}[h]
  \begin{lstlisting}
parse :: Eff (Fail :> (State s :> Void)) a -> s -> (s, Maybe a)
parse p inp = run . runState inp . runFail $ p

parse' :: Eff (State s :> (Fail :> Void)) w -> s -> Maybe (s, w)
parse' p inp = run . runFail . runState inp $ p
  \end{lstlisting}
  \caption{Running parsers}
  \label{listing:extEffparse}
  \end{figure}


\chapter{Design of Markdown parser}

\lstinline{Markdown} is a lightweight language, widely used for small scale 
writing. It comes in handy when regular markup languages such as 
\lstinline{HTML} and~\LaTeX~are considered an overkill.~\lstinline{Markdown}
is popular in IT community, for instance it is extensively used on source code 
repositories hosting web sites, like \lstinline{GitHub}~\cite{github}.

\section{Markdown Syntax}

  In contrast with~\lstinline{HTML} or~\lstinline{XML},~\lstinline{Markdown}
  doesn't have a standard. However, informal but comprehensive description of 
  syntax exists~\cite{markdownSyntax}. There are also several enhanced versions, 
  such as, for example,~\lstinline{GitHub Flavoured Markdown}~\cite{GFM}.

  In this work a subset of~\lstinline{Markdown} syntax is considered, 
  specifically headers, paragraphs, unordered lists and block quotes. In addition, 
  source code may include~\LaTeX-blocks with formulae.   

\section{Parser}

  \lstinline{Haskell} programming language is know for its rich type system. 
  It provides facilities of algebraic data types (ADTs), that could be exploited 
  to conveniently express structure of abstract syntax tree (AST). Every 
  \lstinline{Document} is a list of blocks. Now, \lstinline{Block} is a sum type, 
  which means that each of its data constructors represents some 
  \lstinline{Markdown}-block.     

  \begin{figure}[t]
  \begin{lstlisting}
  type Document = [Block]

  data Block = Blank
             | Header (Int,Line)
             | Paragraph [Line]
             | UnorderedList [Line]
             | BlockQuote [Line]
    deriving (Show,Eq)

  data Line = Empty | NonEmpty [Inline]
    deriving (Show,Eq)

  data Inline = Plain String
              | Bold String
              | Italic String
              | Monospace String
    deriving (Show,Eq)
  \end{lstlisting}
  \caption{Types of \lstinline{Markdown} abstract syntactic tree}
  \label{listing:MarkdownADT}
  \end{figure}

  Let's take a closer look at types from listing~\ref{listing:MarkdownADT}.
  \lstinline{Block} is either empty block, or header, or paragraph, or
  unordered list, or block quote. Most blocks is essentially a list of lines.
  Every line is a collection of inline elements that are treated differently 
  based on its style. Listing~\ref{listing:MarkdownInline} contains parsers for 
  line and inline elements, parsers \lstinline{bold}, \lstinline{italic} and 
  \lstinline{plain} are similar to \lstinline{monospace} and are omitted for 
  the sake of briefness.

  Implementation of \lstinline{Markdown} parsers heavily relies on base of 
  combinators, that have been described in chapter 3. It won't be redundant to
  remind types of those combinators (see listing~\ref{listing:ParserCombinators}).   

  \begin{figure}[h]
  \begin{lstlisting}
  many :: Parser t a -> Parser t [a]

  sepby :: Parser t a -> Parser t b -> Parser t [a]

  bracket :: Parser t a -> Parser t b -> Parser t c -> Parser t b
  \end{lstlisting}
  \caption{Type signatures of base parser combinators}
  \label{listing:ParserCombinators}
  \end{figure}

  \begin{enumerate}
    \item \lstinline{many} parses a list of tokens which satisfy its argument.
    \item \lstinline{sepby} parses a sequence of tokens which satisfy its first 
    argument and separated by tokens which satisfy second one.
    \item \lstinline{bracket} parses tokens which satisfies its third argument 
    and enclosed by tokens which satisfy first and third one respectively.
  \end{enumerate}

  \begin{figure}[t]
  \begin{lstlisting}
  line :: TM.TextualMonoid t => Parser t Line
  line = emptyLine `mplus` nonEmptyLine

  emptyLine :: TM.TextualMonoid t => Parser t Line
  emptyLine = many (sat wspaceOrTab) >> char '\n' >> return Empty

  nonEmptyLine :: TM.TextualMonoid t => Parser t Line
  nonEmptyLine = do
    many (sat wspaceOrTab)
    l <- sepby1 (bold <|> italic <|>
                 plain <|> monospace) (many (char ' '))
    many (sat wspaceOrTab)
    char '\n'
    return . NonEmpty $ l

  monospace :: TM.TextualMonoid t => Parser t Inline
  monospace = do
    txt <- bracket (char '`') sentence (char '`')
    p   <- many punctuation
    return . Monospace $ txt ++ p
  \end{lstlisting}
  \caption{Parsers of inline elements}
  \label{listing:MarkdownInline}
  \end{figure}

  Being able to to correctly parse both lines and inline elements, it's 
  time to get to block parsers. Listing~\ref{listing:MarkdownHeader} contains 
  parser for header and~\ref{listing:markdownUlist} for unordered list. Parsers 
  for the rest of blocks may be constructed in a similar way.

  \begin{figure}[t]
  \begin{lstlisting}
  header :: TM.TextualMonoid t => Parser t Block
  header = do
    hashes <- token (some (char '#'))
    text <- nonEmptyLine
    return $ Header (length hashes,text)
  \end{lstlisting}
  \caption{Parsers for header}
  \label{listing:MarkdownHeader}
  \end{figure}

  \begin{figure}[t]
  \begin{lstlisting}
  unorderedList :: TM.TextualMonoid t => Parser t Block
  unorderedList = do
    items <- some (token bullet >> line)
    return . UnorderedList $ items
    where
      bullet :: TM.TextualMonoid t => Parser t Char
      bullet = char '*' <|> char '+' <|> char '-' >>= return
  \end{lstlisting}
  \caption{Parser for unordered list}
  \label{listing:markdownUlist}
  \end{figure}

  \lstinline{Markdown} language is also used for making notes during lectures 
  and talks, building documentation, and preparing assignments. Therefore, 
  \LaTeX~blocks seem as a helpful enhancement of a language. There almost no 
  additional work to be done here: it's needed to recognize a \LaTeX~block and 
  leave its contents unmodified, so could be later treated properly 
  by code generator.      

  \begin{figure}[h]
  \begin{lstlisting}
  blockMath :: TM.TextualMonoid t => Parser t Block
  blockMath =
    (bracket (string "$$") (some (sat (/= '$'))) (string "$$")) >>=
    return . Paragraph . (: []) . NonEmpty . (: []) . Plain .
      (\x -> "$$" ++ x ++ "$$")
  \end{lstlisting}
  \caption{Parser for \LaTeX-blocks}
  \label{listing:MarkdownLaTeX}
  \end{figure}

  Listing \ref{listing:MarkdownDoc} presents top-level parser for 
  \lstinline{Markdown}-document as a list of blocks.

  \begin{figure}[h]
  \begin{lstlisting}
  doc :: TM.TextualMonoid t => Parser t Document
  doc = many block
    where block = blank <|> header <|> paragraph <|>
                  unorderdList <|> blockquote <|> blockMath
  \end{lstlisting}
  \caption{Top-level parser for \lstinline{Markdown}-document}
  \label{listing:MarkdownDoc}
  \end{figure}

\section{HTML generation}

  Having an AST, code in any markup language could be generated. In this work, 
  \lstinline{HTML} has been chosen as a target language. One advantage of \lstinline{HTML} 
  is possibility of use of \lstinline{JavaScript}-libraries,
  such as \lstinline{MathJax}~\cite{mathJax} to render \LaTeX~blocks. 

  Code generation process follows structure of abstract syntactic tree: function
  \lstinline{serialize} generated code for list of blocs and collapses result to
  a single string. Every block type is handled by separate pattern matching clause 
  of~\lstinline{genBlock} function. Equally for lines elements and function 
  \lstinline{genLine}.

  Listing~\ref{listing:HTMLGen} displays simplified code generators: handlers 
  for some items are omitted for compactness.  

  \begin{figure}[h]
  \begin{lstlisting}
  serialize :: Document -> String
  serialize = concatMap genBlock

  genBlock :: Block -> String
  genBlock Blank = "\n"
  genBlock (Header h) =
    "<h" ++ s ++ ">" ++ genLine (snd h) ++ "</h" ++ s ++ ">" ++ "\n"
      where s = show (fst h)
  genBlock (UnorderedList l) =
    "<ul>" ++ concatMap ((++ "\n") . genOrderedListItem) l ++ "</ul>" ++ "\n"

  genLine :: Line -> String
  genLine Empty         = ""
  genLine (NonEmpty []) = genLine Empty ++ "\n"
  genLine (NonEmpty l)  = concatMap ((++ " ") . genInline) l

  genOrderedListItem :: Line -> String
  genOrderedListItem l = "<li>" ++ genLine l ++ "</li>"

  genInline :: Inline -> String
  genInline (Plain s) = s
  genInline (Monospace s) = "<code>" ++ s ++ "</code>"
  \end{lstlisting}
  \caption{\lstinline{HTML} markup generation}
  \label{listing:HTMLGen}
  \end{figure}

  This is, in brief, the process of~\lstinline{Markdown} parsing and \lstinline{HTML}
  code generation. Full source codes of parsers and code generator may be found 
  in~\lstinline{GitHub} repository~\cite{mdParse}.


\chapter{Conclusion}

\section{Main results achieved} 
  
  Following \textbf{results} have been achieved:

  \begin{enumerate}

    \item Parser combinator library based on monad transformers that uses special 
    monoids for input stream representation has been developed. Source code of
    library is available on GitHub~\cite{mdParse}.

    \item Prototype of parser combinator library based on extensible effects has
    been developed. Source code of library is available on GitHub~\cite{mdParse}.

    \item Basing on library from point one, parser for subset of Markdown enriched 
    with \LaTeX blocks has been built, together with~\lstinline{HTML} code generator. Source code of implementation and build instructions are also available on 
    Github~\cite{mdParse}.   

  \end{enumerate}

  In addition, section 1.2 contains a comparative analysis of convenience of 
  programming with two approaches to control of computational effects: monad
  transformers and extensible effects.

\section{Possible applications}

  Developed libraries may be used for syntax analysis of markup and programming 
  languages.

  One possible application of Markdown with \LaTeX blocks parser is a electronic 
  lecture notes system. 

\section{Future research}
  
  Extensible effects is a implementation of abstractions of algebraic effects and 
  effects handlers. These abstractions are in its infancy and it could be useful 
  to perform an approbation of its implementations as a machinery for constructing  
  parser combinators libraries.

% Печать списка литературы (библиографии)
\printbibliography[heading=bibintoc%
    %,title=Библиография % если хочется это слово
]
% Файл со списком литературы: biblio.bib
% Подробно по оформлению библиографии:
% см. документацию к пакету biblatex-gost
% http://ctan.mirrorcatalogs.com/macros/latex/exptl/biblatex-contrib/biblatex-gost/doc/biblatex-gost.pdf
% и огромное количество примеров там же:
% http://mirror.macomnet.net/pub/CTAN/macros/latex/contrib/biblatex-contrib/biblatex-gost/doc/biblatex-gost-examples.pdf

\end{document}
