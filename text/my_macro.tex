

% макросы для начала введения и заключения
\newcommand{\Intro}{\addsec{Введение}}
\ifthenelse{\value{worktype} > 1}{%
    \renewcommand{\Intro}{\addchap{Введение}}%
}

\newcommand{\Conc}{\addsec{Заключение}}
\ifthenelse{\value{worktype} > 1}{%
    \renewcommand{\Conc}{\addchap{Заключение}}%
}

% Правильные значки для нестрогих неравенств и пустого множества
\renewcommand {\le} {\leqslant}
\renewcommand {\ge} {\geqslant}
\renewcommand {\emptyset} {\varnothing}

% N ажурное: натуральные числа
\newcommand {\N} {\ensuremath{\mathbb N}}

% значок С++ — используйте команду \cpp
\newcommand{\cpp}{C\nolinebreak\hspace{-.05em}%
\raisebox{.2ex}{+}\nolinebreak\hspace{-.10em}%
\raisebox{.2ex}{+}}

% Неразрывный дефис, который допускает перенос внутри слов,
% типа жёлто-синий: нужно писать жёлто"/синий.
\makeatletter
    \defineshorthand[russian]{"/}{\mbox{-}\bbl@allowhyphens}
\makeatother
